\chapter{はじめに}

集合論は数学の基礎である.
現代の数学理論の中で,
集合論の基本的な知識なしに学べる理論は
皆無であると言ってもいい.
しかし,これは集合論について書かれた専門書を
完璧に読破しなければ
他の分野に移れないという意味ではない.
たとえば,集合論の講義よりも先に
微分積分学や線形代数学の講義が
行われている理工系の大学は決して珍しくはない.
それらの講義では,集合や写像の説明が
あくまで補足として述べられることが多い.
これらの分野でももちろん集合や写像といった概念
は登場するが,それはあくまで「ことば」としてであって,
集合論の本に載せられている定理を
空気や水のように使っていくというものではない.
それでも,集合論で扱う概念が
「ことば」としてまるで空気や水のように
扱われていることには違いない.


ところで,集合とはいったいどのような概念であろうか.
本書では「集合とは,範囲の確定したモノの集まりである」
として集合という概念を導入した.
ではこれが集合という概念の「本質」であるかと
言われれば,少し怪しいように感じられる.
というのも,「数を一列に並べたもの」と解釈される
数列も,「数と数との対応」と解釈される関数も,
すべて集合として定式化できるからである
\footnote{%
  通常,数列や関数は一種の写像として定式化されるが,
  写像もやはり集合のみを使って定式化される.
  従って,「集合はモノの集まりで,
  写像は集合と集合の対応だから,
  数列や関数が集合というのは間違いだ」
  という批判は的外れである.
}%
.


では,集合の「本質」とはいったい何なのだろうかと
考えてみると,むしろそのような問は,
少なくとも数学を学ぶという目的に限っては
あまり価値はないのではないかと思われる.
というのも,たとえ集合の「本質」が
何か規定されたとしても,
それによって既存の数学理論から一切集合が消滅したり,
あるいは集合の扱われ方が劇的に変わったりすることは
おそらくないからである.
そのような多くの場合空虚で陳腐になりがちな
問に対する答えを追いかけるよりも,
集合論が見せる豊かな世界を探求するほうが
よほど建設的である.
集合の「本質」を探求することに
まったく価値がないとも思えないが,
よく考えないままそのような探求に邁進すると
「あらゆる概念にはアプリオリに本質なるものが
一意に備わっており,それらをひとつひとつ
理解していくのが学ぶということである」
という価値観にのめり込んでしまうこともあり,
こうなってしまうと
数学の学習に対して大きな悪影響を及ぼしてしまう.


最初に述べたように,
集合論で扱う概念は多種多様な分野で
いろいろな使われ方をしており,
そこに「たったひとつの本質」を見出すことは
ほとんど不可能である.
集合論で扱うものだけではなく
初学者が出くわす概念のほとんどがそうである.
「多項式とはいったい何なのか」やら
「掛け算とはいったいなのか」
と言われても,
質問者が期待する答えを返すのは
無理だろうと思う
\footnote{%
  時折ある概念のひとつの側面でしか
  ない例を持ち出し,
  「これがこの概念の本質だ」
  などと主張するのを見かけることがあるが,
  これはあまり褒められたものではないと思う.
  たとえば「行列は線形変換だ」
  などという主張はよく聞くものである.
}%
.


本書は,現代数学の「ことば」としての
集合論が扱えるようになることを目標とした
本である.
第\ref{chp:sequent}章で
前提知識として必要となる
数学で扱う論理に関して解説する.
このあたりの話題はそもそも何を目的としているのか
すらわかりにくいことがある.
そのため,第\ref{chp:gensuu}章で
歴史的な背景を解説することにした.
本格的に集合が登場するのは
第\ref{chp:set}章からである.
第\ref{chp:set}章では,
集合と写像に関する基本的な事項を解説する.
第\ref{chp:realnumber}章では,
Dedekind切断を用いて有理数から
実数を構成し,
その性質を調べていく.
この章での話題は集合論から続く位相空間論の
イントロダクションとなるものである.
第\ref{chp:cardinal}章では,
集合論が数学の一分野としての地位を確立する
きっかけとなった
濃度に関する議論を行う.
無限集合や選択公理に関する議論に触れることで,
集合論の奥深さが実感できるのではないかと思う.

また,多くの入門書では
前提知識となる論理に関する
話題は
真理値を用いた意味論的な議論で
解説していることが多いのだが,
本書ではシークエント計算を用いた
統語論的な議論で解説することにした.
その目的としては,
数学における(数式がほとんど出てこない)
証明に慣れること,
そして「証明」という検証作業によって
結果の「正しさ」に納得することが挙げられる
\footnote{%
  もちろん「証明できる」ことと「正しい」
  ということが本当に等価なのであるかということは
  大問題である.
}%
.
このあたりの議論を本格的にやろうとすると
メタ理論としての集合論が必要になり,
集合論の入門書としては手に負えなくなってしまう.
そのため,ほとんどの箇所を素朴な議論で済ませている.
議論に不満を感じた人は,
ぜひ参考文献にある数理論理学の本を手にとり,
さらに学習を進めてもらいたい.


本書を読む際には,
ぜひ「書いてある内容を書き写す」
という作業を面倒くさがらずにやってほしい.
もちろん,書かれている文字列を
一字一句書き写すことには
まったく意味がない.
あくまで書き写すべきは「書いてある内容」である.
自分が読んだ内容を自分一人の手で
再構築することで,
その内容が自分の血肉となり,
頭に染み付くのである.










第2版にあたり,誤字脱字を訂正し,
Zornの補題の応用に関して少々の話題を追加したほか,
誤解の多い記述を修正・削除した.
これで,一般的な集合・位相の教科書に
載せられている集合論の内容は
ほぼ網羅できたのではないかと思う.


おわりに,本書がこれから数学を学ぼうとする人に対して
少しでも力添えできれば何よりである.
\begin{flushright}
  2018年12月 著者
\end{flushright}
