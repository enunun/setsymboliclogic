\chapter{現代数学への展望}
\label{chp:gensuu}
 本書を手に取った読者の中には初めて数学を本格的に学ぼうとする人も多いだろう.
 そのような人の中には,現代数学は複雑怪奇でよくわからない世界という
 イメージを持っている人も多いと思う.
 特に,「公理とは,絶対普遍の真理である」などと考えている人も多いのではないだろうか.

 この章では,そうした「公理」や「定義」などの数学を学ぶ上では絶対に知っておかなければならない
 基本的な用語に関してざっくりと解説する.

 この章では,本書でのちに学ぶ概念や他の分野(特に代数学)で
 学ぶ事項のうちで既知としたものがある.
 従って,最初にこの章を読んでから他の章に移るのではなく,
 実際に数学を学びながら暇をつぶす感覚でこの章を読むとよい.

 \newpage

 \section{幾何学の歴史と公理論}
 \label{sec:kikakouri}
  公理論的な現代数学の価値観は,幾何学とともに形成されていったといっても過言ではない.
  ここでは,幾何学の歴史をおおざっぱに見ていこう.

 \paragraph{幾何学の起源}
  幾何学は,紀元前に古代エジプトにおいて測量の必要性から生まれた.
  しかし,そこでの「幾何学」は現代のように演繹的証明を積み重ねていくものではなかった.
  それは幾何学だけではない.当時の古代エジプトでは十進法や単位分数,
  辺の長さの比が$3:4:5$の直角三角形などが発見されていたが,
  その特徴は「具体的な例題の解法の習得による学習」であった.

  また,紀元前の幾何学といえば古代バビロニアも見逃せない.
  古代バビロニアでは六十進法が主として用いられた.
  他にも,2次方程式の解法やピタゴラスの定理についての研究もなされていたようである.
  しかし,その特徴は古代エジプトと同じく「具体的な例題の解法の習得による学習」であった.

 \paragraph{古代ギリシャの数学}
  古代エジプトと古代バビロニアでの幾何学に共通するのは
  「具体的な例題の解法の習得による学習」である.
  すなわち,古代エジプトと古代バビロニアで培われていた知識はどちらも
  実用性を重視したものであったと考えることができる.
  ところが,古代エジプトと古代バビロニアで培われていた知識は古代ギリシャに渡り,
  そこで大きな変貌を遂げることとなった.
  それは,実用的知識をある理論体系にまとめあげ,そこから得られた結果を
  再び具体的な問題に当てはめるというものである.
  つまり,古代ギリシャでは「具体的な対象から推測される
  一般化された主張に証明を与え,それを再び具体的な対象に適用する」
  という形式で幾何学が展開されていったのである.

  \index[nidx]{Thales@Thales(ターレス)}
  Thales(624 B.C.頃--547 B.C.頃)はギリシャ哲学の祖とも言われており,
  上記のような考えのもと,さまざまな主張を一般化し,その証明を与えていった.
  以下にThalesの発見と言われていることがらをいくつか挙げておこう.
  \begin{itemize}
    \item 円は直径により2等分される.
    \item 二等辺三角形の両底角は相等しい.
    \item 2直線が交わるとき,その対頂角は相等しい.
    \item 2つの三角形において,一辺とその両端の角がそれぞれ等しければ,
      2つの三角形は合同である.
    \item 1つの円において,直径に対する円周角はつねに直角である.
  \end{itemize}

 \paragraph{Euclidの原論}
  古代ギリシャにおいては幾何学の研究は以後も続けられていたが,
  \index[nidx]{Euclid@Euclid(ユークリッド)}
  古代ギリシャの数学者Euclid(300 B.C.頃)の著書『原論』により,
  それまでに得られていた数学研究の成果が演繹的体系としてまとめ上げられた.

  『原論』では,まずこれから用いる「点」や「直線」,あるいは「面」などの
  言葉の定義がなされており,
  ついで幾何学を展開するときの基礎となる5つの公準(要請)と
  数学全体において基礎となる公理(共通概念)が述べられている.
  その後,用意した公理と公準に基づき,演繹的論証を積み重ねることによって
  当時知られていたいくつもの定理が証明されている.
  しかし,『原論』における「公理」とは「一般に共通する普遍の真理」であり,
  「公準」とは「幾何学における要請」であったことに注意しておかなければならない
  \footnote{ここでの「公準」は「幾何学における公理」であると思っておけばよい.}
  .
  すなわち,『原論』においては
  「公理」は「自明の事実」として扱われていたのである.

  『原論』は紀元前に書かれた幾何学の教科書である.
  しかし,後世の人々によって加筆されたり翻訳されたりなどにより,
  長い間幾何学の標準的な教科書として使われ続けていた.
  それほどまでの完成度の教科書が紀元前の時点ですでに出現していたことは
  驚くべきことであろう.

  『原論』の冒頭で定義されている用語は23個あるが,そのうちのいくつかを見てみよう.
  \begin{itemize}
    \item 点とは部分を持たないものである.
    \item 線とは幅のない長さである.
    \item 線の端は点である.
    \item 直線とはその上にある点について一様に横たわる線である.
    \item 平面とはその上にある直線について一様に横たわる面である.
  \end{itemize}
  この「定義」についてはのちにもう少し深く考えることにして,
  『原論』における公準と公理も見ておこう.
  『原論』における公準は以下の通りである.
  \begin{enumerate}[公準1.]
    \item 任意の点とこれと異なる他の任意の点とを結ぶ直線を引くことができる.
    \item 任意の線分はこれを両方にいくらでも延長することができる.
    \item 任意の点を中心とする任意の半径の円を描くことができる.
    \item 直角はすべて互いに等しい.
    \item 2直線が1直線と交わるとき,その同じ側にできる内角の和が
      2直角よりも小さいならば,2直線はその側に延長すると必ず交わる.
  \end{enumerate}
 そして,『原論』において述べられている公理は以下の通りである
 \footnote{これらの公理はすべて妥当であるかのように思えるが,実は第5公理は現代の
 数学においては(ある意味で)正しくないことが\ref{sec:aleph}で明らかとなる.}
 .
  \begin{enumerate}[公理1.]
    \item 同一のものに等しいものはまた互いに等しい.
    \item 等しいものに等しいものを加えれば,その結果もまた等しい.
    \item 等しいものから等しいものを引けば,その結果もまた等しい.
    \item 互いに重なり合うものは互いに等しい.
    \item 全体は部分より大きい.
  \end{enumerate}
  


  先に述べた公準と公理を眺めていると,
  明らかに第5公準だけが複雑で長ったらしい主張であることに気づく.
  この第5公準は他の公準から導けるのではないかという疑問が出現し,
  多くの数学者がその証明に取り組んだ.
  しかし,いくら探しても第5公準が他の公準から証明されることはなかった.
  

 \paragraph{非Euclid幾何学の登場}
  この問題に解決の兆しが見られたのは19世紀に入ってからのことである.
  \index[nidx]{Gauss@Gauss(ガウス)}
  Gaussや
  \index[nidx]{Lobachevski@Lobachevski(ロバチェフスキー)}
  Lobachevski,
  \index[nidx]{Bolyai@Bolyai(ボヤイ)}
  Bolyaiらの手により,第1公準から第4公準を仮定し,第5公準を否定する幾何学が
  構築された
  \footnote{Gaussは,宗教的な論争に巻き込まれることを恐れたのか公表はしていない.
  しかし,このような幾何学が存在するということは確信していたようである.}
  .
  このことから,第1公準から第5公準までを
  すべて仮定するような幾何学は
  \index[widx]{Euclidきかがく@Euclid幾何学}Euclid幾何学と呼ばれるようになり,
  第5公準を否定するような幾何学は
  \index[widx]{Euclidきかがく@Euclid幾何学!ひEuclidきかがく@非---}非Euclid幾何学
  と呼ばれるようになった.

  こうしてEuclid幾何学と非Euclid幾何学という2つの幾何学が提案されたわけであるが,
  Euclid幾何学や非Euclid幾何学の公理系
  \footnote{ここではとりあえず「公理の集まり」と考えておくとよい.
  また,ここでの「公理」は『原論』でいうところの「公準」に相当する.}
  そのものが
  矛盾を抱えていないということが示されたわけではないということに注意しなくてはならない.
  矛盾が生じるような公理を仮定しても
  そこから構築される理論にあまり意味はない.
  また,一見仮定した公理系に矛盾がないように見えても,
  そこから矛盾した結果が導かれることはないという保証はない.
  公理系の無矛盾性は決して自明なことではないのである.

  Euclid幾何学や非Euclid幾何学の公理系の無矛盾性に関しては,
  \index[nidx]{Klein@Klein(クライン)}
  Kleinや
  \index[nidx]{Beltrami@Beltrami(ベルトラミ)}
  Beltramiらの手により不完全ながらも解決されることとなった.
  その方法は「Euclid幾何学の公理系で許された手法のみを利用して,
  非Euclid幾何学の公理系を満たす対象を実際に構成してみせる」
  というものである.
  ある公理系を満たす対象のことをその公理系の
  \index[widx]{こうり@公理 \, axiom!もでる@---系のモデル \, model} 
  \emph{モデル}(model)と呼ぶ.
  すなわち,KleinやBeltramiらが行ったのは
  Euclid幾何学上に非Euclid幾何学のモデルを構築したということである.
  非Euclid幾何学のモデルとして有名なのは
  \index[nidx]{Riemann@Riemann(リーマン)}
  Riemannによって構築された球面幾何学であろう.
  それらの非Euclid幾何学の詳細は省くが,
  非Euclid幾何学の公理系を満たす対象が具体的に構成されたからには,
  非Euclid幾何学の公理系の無矛盾性を認めざるを得ない.
  「正しい」手続きのもとで構成された対象が
  満たす性質に矛盾が存在するわけがないからである.
  しかし,非Euclid幾何学のモデルはEuclid幾何学のもとで構築されたわけであるから,
  実際にいえるのは「Euclid幾何学の公理系が無矛盾であれば,
  非Euclid幾何学の公理系も無矛盾である」ということである.

  非Euclid幾何学の登場により,「幾何学」はEuclid幾何学だけというわけではなく,
  異なる公理系を採用した別の幾何学も存在しうることが明らかとなった.
  すると,「そもそも幾何学とは何なのか?」という疑問が出てくる.
  この問に対して1つの答えを与えたのがKleinである.
  Kleinは「空間
  \footnote{空間に関しては\ref{sec:structure}で解説する.変換群に関しては代数学の本を参照せよ.}
  $S$と$S$上の変換群$G$が与えられたとき,
  $S$の部分集合,すなわち$S$上の図形に関する種々の性質や量のうち,
  $G$に属するすべての変換によって不変に保たれるものを研究する分野が
  $G$に従属する$S$の幾何学である」
  とした.
  この主張はKleinが1872年にエルランゲン大学の教授職に就くときに示されたものであり,
  \index[widx]{えるらんげんぷろぐらむ@エルランゲン・プログラム}エルランゲン・プログラム
  と呼ばれている.
  エルランゲン・プログラムの登場により,Euclid幾何学や射影幾何学といった
  当時知られていた様々な幾何学を統一的な視点で考察することが可能となった.

  ある変換で不変な性質や量に研究対象を絞るということは,
  その変換で移り合う図形は「同じ図形」であるとみなすことに相当する.
  従って,幾何学は
  \begin{itemize}
    \item どのようなものを図形と考え,
    \item どのような図形を「同じ図形」とみなすか
  \end{itemize}
  によって決まると考えることができる.

  エルランゲン・プログラムに従うのであれば,
  Euclid幾何学とは「Euclid空間という空間において,
  合同変換と呼ばれる平行移動,回転,鏡像変換などの変換
  で不変な性質や量を研究する分野」と考えることができる.
  Euclid幾何学における主な研究対象として線分の長さや面積,あるいは角度
  などが挙げられる.
  これらの量は図形を平行移動したり回転したりしても不変に保たれることは
  容易にわかるだろう.

 \paragraph{公理論と形式主義}
  さて,Euclid幾何学の公理系が無矛盾であるのならば
  非Euclid幾何学の公理系も無矛盾であるのであった.
  ではEuclid幾何学の公理系は無矛盾であるのか? と考えるのは当然である.
  この問に対しては,そもそもEuclid幾何学の公理系に
  論理的な不備があったことに注目しなくてはならない.
  
  \index[nidx]{Hilbert@Hilbert(ヒルベルト)}
  Hilbertは1899年に著書『幾何学の基礎』において,
  Euclid幾何学の論理的不備を正した公理系を提出した
  \footnote{現代の視点で述べるのであれば,Hilbertが提案したEuclid幾何学の公理系は
  少々扱いづらく,
  \index[nidx]{Tarski@Tarski(タルスキ)}
Tarskiという人物が考案した公理系のほうが便利である.}
  .
  『幾何学の基礎』においては,「点」や「直線」などの用語が
  使われているにも関わらず,その定義は書かれていない.
  これは『原論』とは大きく異なる点である.
  また,公理中で「…が---の間にある」などといった言葉が使われているが,
  これにも定義は書かれていない.
  これは,Hilbertが定義する必要のない自明な概念として用いたからではない.
  これらの用語はそもそも定義なしに用いる用語であるとしたのである.
  『原論』では,「点とは部分を持たないものである」だとか「線とは幅のない長さである」
  などといった定義がなされているが,この「部分をもたない」や「幅のない長さ」
  とは一体なんだろうかということを考えてみると,
  どういうように定義をしても結局は循環論法になってしまうことに気づく.
  そこで,Hilbertはこれらの用語を定義することなしに公理を説明した.
  このように,公理を説明するために定義せずに用いる用語を
  \index[widx]{むていぎじゅつご@無定義術語 \, undefined term}
  \emph{無定義術語}(undefined term)という.
  『幾何学の基礎』においては,
  「点」や「直線」,「平面」といった用語が無定義術語である.
  定義されないのだから,これらの用語を別のものに置き換えても理論に何ら支障はない.
  すなわち,これらの用語を「テーブル」,「椅子」,「ビールジョッキ」
  といった言葉に置き換えても理論としてはまったく同じであるとしたのである.
  この頃から,
  \index[widx]{こうり@公理 \, axiom}
  \emph{公理}(axiom)とは,
  理論の出発点となる単なる仮定であり,
  それが自明であるなどとは考えないようになった.
  この考え方に基づいて組み立てられた理論は
  \index[widx]{こうり@公理 \, axiom!こうりろん@---論 \, axiomatics}
  \emph{公理論}(axiomatics)的であるという
  \footnote{よく
    \index[widx]{こうり@公理 \, axiom!こうりしゅぎ@---主義 \, axiomatism}
    \emph{公理主義}(axiomatism)
    という言葉が使われることがあるが,この言葉は日本だけでしか使われないようである.
    しかし,Hilbertの思想を表現するには便利なのでしばしば使われる.
    もちろんaxiomatismも和製英語である.}
  .しかしHilbertはさらに,公理からの「演繹的推論」にも手を出した.
  Hilbertは我々が当たり前のように使っている
  演繹的推論も定式化して議論できるのではないかと考えたのである.
  Hilbertは,公理は「意味」を
  
  持たない単なる記号列であり,
  そこから定められたルールに基づく記号操作により,
  あらゆる命題が導かれるとした.
  そして,この「記号操作」こそが
  \index[widx]{しょうめい@証明 \, proof}
  \emph{証明}(proof)であるとしたのである.
  Hilbertのこの考え方は
  \index[widx]{けいしきしゅぎ@形式主義 \, formalism}
  \emph{形式主義}(formalism)と呼ばれている.
  本書ではHilbertが考案した演繹体系は取り扱わないが,
  代わりに第\ref{chp:sequent}章において,
  \index[nidx]{Gentzen@Gentzen(ゲンツェン)}
  Gentzenが考案した自然演繹という演繹体系
  (をシークエント計算という手法を使ってとっつきやすくしたもの)を取り扱う.
  自然演繹は形式主義に基づく演繹体系の1つである.

  理論の出発点として仮定した公理系には矛盾があってはならない.
  また,仮定した公理系に他の公理から導かれる結果が混ざっているのは美しくない.
  すなわち,公理系は全体として矛盾を含まないという
  \index[widx]{むむじゅんせい@無矛盾性 \, consistency}
  \emph{無矛盾性}(consistency)は絶対に必要であり,
  さらに,他の公理から導かれる結果が混ざっていないという
  \index[widx]{どくりつせい@独立性 \, independence}
  \emph{独立性}(independence)があるのが望ましい.

  さて,Hilbertの功績により,
  Euclid幾何学の公理系の無矛盾性が本格的に議論できるようになったわけであるが,
  その問題はすでに解決していたといってもよい.
  \index[nidx]{Descartes@Descartes(デカルト)}
  Descartesが創始した解析幾何学がEuclid幾何学のモデルになっていると考えられるからである.
  解析幾何学は実数論に基づいて構築されているから,
  このことからわかるのは「実数論の公理系が無矛盾であるならば,
  Euclid幾何学の公理系も無矛盾である」ということになる.
  実数論の無矛盾性に関しては,
  第\ref{chp:realnumber}章においてそのモデルを構成することを試みる.
  
  公理論的な数学理論の作り方をまとめておこう.
  まず,基礎となる用語や概念を無定義術語として用意する.
  そして,理論の基礎となる命題を公理に据える.
  公理を仮定して論理や集合を用いて証明された命題を
  \index[widx]{ていり@定理 \, theorem}
  \emph{定理}(theorem)と呼ぶ
  \footnote{通常は定理の中でも特に重要なもののみを「定理」として挙げる場合が多い.
  また,「定理」の意味で「命題」という言葉が使われることも多い.}
  .もちろん,すでに得られた定理を利用することで新たな定理を得ることも許される.
  また,定理の中でもある定理から即座に得られるものはその定理の
  \index[widx]{けい@系 \, corollary}
  \emph{系}(corollary)と呼び,
  その定理自体にはさほど興味がなく,
  そこから導かれる結果のほうに関心があるとき,
  もととなるその定理のことを
  \index[widx]{ほだい@補題 \, lemma}
  \emph{補題}(lemma)と呼ぶことがある.
  定理,系,補題には明確な区別があるわけではなく,
  どの用語が使われるかは理論を構築する人の意図による.
  また,よく使う対象や関係を一言で表したいときには,
  それらの対象や関係に名前をつけて活用すればよい.
  名前のついた対象や関係は
  \index[widx]{ていぎ@定義 \, definition}
  \emph{定義}(definition)されているという.

  もう少し形式的に述べておこう.
  まず,どのような理論を構築したいかに関わらず,
  等号「$=$」や「$\land$」や「$\lnot$」などの論理記号,
  「$x,y, \ldots$」などの変数記号,あるいはカッコやカンマ
  などの記号はあらかじめ用意されていなくてはならない.
  また,これらの記号は理論を構築する前に用意するため,
  変数記号はいくつ必要になるかわからない.
  そのため,変数記号は無限に多く用意されているものとしよう.
  以上の準備のもと,理論を構築することを考える.
  理論(theory)は,以下の手順で構築することができる:
  \begin{enumerate}[1. ]
    \item 理論を記述するのに必要な記号がどのようなものであるかを決める.
      これを言語(language)と呼ぶ.
    \item 1.で決定した言語$L$を用いていくつかの文(sentence)をつくり,
      これを公理系とする.
  \end{enumerate}
  理論とは,2.で作った文の集まり,すなわち公理系のことである.
  いくつかの理論の例を見てみよう.

  \begin{ex}[群の理論]
    群の理論における言語$L$は$L= \Set{ e, \ast , {}^{-1} }$
    と与えられ,群の理論$T$は以下の3つの文からなる:
    \begin{enumerate}[G1. ]
      \item $\forall x,y,z ((x \ast y) \ast z = x \ast (y \ast z ) ),$ 
      \item $\forall x ( x \ast e = e \ast x = x ) , $
      \item $ \forall x ( x \ast x^{-1} = x^{-1} \ast x = e ).$
    \end{enumerate}
    群の理論のモデルとして,
    たとえば$(\mathbb{Z} , 0, +, -)$が挙げられる.
    $0$や$+$や$-$はそれぞれ通常の整数$0$と
    加法,減法を表す.
  \end{ex}


  \begin{ex}[順序の理論]
    順序の理論における言語$L$は$L= \Set{ \leq }$と与えられ,
    順序の理論$T$は以下の3つの文からなる:
    \begin{enumerate}[O1. ]
      \item $\forall x (x \leq x) ,$
      \item $\forall x,y ({x \leq y} \land { y \leq x } \to x = y) ,$
      \item $\forall x,y,z ({x \leq y } \land {y \leq z} \to x \leq z).$
    \end{enumerate}
    順序の理論のモデルとしては,たとえば$( \mathbb{N} , {\leq})$
    が挙げられる.
  \end{ex}

  群の理論における$e$や${\ast},{}^{-1} $といった記号は
  他の変数記号は違った「特別な役割」が与えられている.
  順序の理論における$\leq$も同じである.
  
  言語をあらかじめ決めておかないと,
  変数記号とこれらの「特別な役割」をもった記号とが
  混同されることを防ぐため,理論を考えるときには
  まずその理論で用いる言語を決定する必要がある.
  しかし,証明能力や無矛盾性を議論したいような文脈でない限り,
  すなわち我々が素朴的な意味で議論をする場合には
  ちょっと気をつけてさえいればあまり問題はない.
  そのため,「いま記述したい理論に必要な言語は何か」
  などということは議論の土台にすら上がらないことも多い.
  そんなことをしなくても誰も困らないからである.


  

  


  




 \section{定義の手法とそのwell-defined性}
 \label{sec:welldef}
  数学に限らず,ありとあらゆる学問分野においてそこで用いられる言葉の定義がなされ,
  それに基づいた議論が行われている.
  数学においても同じだが,少し特殊な方法で定義がなされることもある.
  後半では定義をする上で保証しなくてはならない
  well-defined性についても解説する.

 \paragraph{定義の手法}
  数学において何かを定義するとき,
  もっともよく使われる手法が「すでに定義された対象や関係,
  もしくは無定義術語の組み合わせでできる
  対象や関係に名前をつける」というものである.
  \begin{ex} \label{ex:defhutuu}
    「各辺の長さがすべて等しい三角形を正三角形という」という定義は,
    「辺」,「長さ」,「三角形」という対象と「等しい」という関係を組み合わせてできる
    「各辺の長さがすべて等しい三角形」という対象に対して
    「正三角形」という名前をつけている.
    また,「整数$a,b$に対し,$a$は$b$で割り切れる,もしくは$b$は$a$を割り切るとは,
    $a=bc$となる整数$c$が存在することをいい,このことを
    $b \mid a$と表す」という定義は,
    整数$a,b$に対する「$a=bc$となる整数$c$が存在する」という関係に
    「$a$は$b$で割り切れる」と「$b$は$a$を割り切る」という2つの名前と
    「$b \mid a$」という記号列を与えている.
  \end{ex}

  定義を書くときには「…を---という」か「---であるとは…であることをいう」
  のどちらかの形式で書くことが多い.「定義する」ということが伝わればよいので,
  その書き方は様々である.

  対象を定義するとき,「${:=}$」や「$\overset{\mathrm{def}}{=}$」あるいは
  「$\equiv$」などの記号を用いて
  \begin{align*}
    f(x) := {x^2 + 1} , \quad 
    f(x) \overset{\mathrm{def}}{=} x^2 + 1 , \quad
    f(x) \equiv x^2+1
  \end{align*}
  などと表すこともある.これらはすべて
  「$f(x)=x^2+1$と定める」という意味である.
  また,関係を定義をするとき,記号「$\Longleftrightarrow$」を用いて
  「整数$n$に対し,
  \begin{align*}
    n \text{が偶数である} \Longleftrightarrow n \text{が2で割り切れる}
  \end{align*}
  と定める」と書かれることもある.
  最後の「と定める」は重要である.
  これがないと定義しているのか事実を述べているのかがわからないからである.
  このように定義する場合,
  \[
    \text{(新たに定義する関係)} \Longleftrightarrow \text{(既知の関係)}
  \]
  という形式で書かれることが多い.

  定義の手法でよく使われるものがもう1つある.
  それは「定義する対象や関係を構成する方法を提示する」というものである.
  このような定義を %
  \index[widx]{ていぎ@定義 \, definition!さいきてきていぎ@再帰的---|see{帰納的定義}}
  \emph{再帰的定義} もしくは
  \index[widx]{ていぎ@定義 \, definition!きのうてきていぎ@帰納的--- \quad recursive ---}
  \emph{帰納的定義}(recursive definition)という.
  \begin{ex} \label{ex:defkinou}
    命題論理式を以下に述べるように帰納的に定義する:
    \begin{enumerate}[(1) ]
      \item 命題変数は命題論理式である.
      \item $\curlyvee$と$\curlywedge$は命題論理式である.
      \item $\varphi$が命題論理式であるとき,
        $(\lnot \varphi)$は命題論理式である.
      \item $\varphi , \psi$が命題論理式であるとき,
        $(\varphi \land \psi) , (\varphi \lor \psi) 
        , (\varphi \to \psi) , 
        (\varphi \rightleftarrows \psi)$
        はすべて命題論理式である.
      \item 以上の規則を有限回適用して得られるもののみが命題論理式である.
    \end{enumerate}
  \end{ex}

  例\ref{ex:defkinou}では,「命題論理式」という用語を
  \begin{enumerate}[1. ]
    \item 出発点となる命題論理式を用意する.
    \item すでに得られた命題論理式から別の命題論理式をつくる方法を述べる.
    \item 以上の操作を有限回適用して得られるもののみが命題論理式であると宣言する.
  \end{enumerate}
  という手順に従って定義している.
  3.のステップはどこまでが命題論理式なのかを指定する重要な
  ステップであるが,明らかであるとして省略されることも多い.

  帰納的定義が使われるのは,定義したい対象を具体的に書き下すことが困難である場合が多い.
  作り方は述べることはできるが作ったものを具体的な式として書き下すのは
  困難であるというのは数学ではよくあることである.

  \begin{ex} \label{ex:zenkasiki}
    数列$\{ a_n \}$を漸化式
    \begin{align*}
      a_1 & = 1, \\
      a_{n+1} & = \sqrt{ a_n + 6} \quad ( n \in \mathbb{N} ) 
    \end{align*}
    で帰納的に定義する.
  \end{ex}

  例\ref{ex:zenkasiki}では,$a_1$がまず与えられ,そこから$a_2 , a_3 , \ldots$
  が順にすべて得られる等式が提示されている.
  この等式により,数列$\{ a_n \}$の任意の項を求めることができる.
  従って,この漸化式は確かに「数列$\{a_n \}$」を定義しているといえる.

  例\ref{ex:defkinou}と例\ref{ex:zenkasiki}からもわかるように,
  帰納的定義においては定義の中に定義したい対象や関係が登場しているという特徴がある.

 \paragraph{well-defined性}
  定義というのは対象や関係に関する単なる約束事であるから
  (一般によく使われるものを除けば)
  何にどんな名前をつけるかは個々人の自由である.
  しかし,定義であればどんなものでも許されるわけではない.
  \begin{ex} \label{ex:welldef1}
    「実数$\alpha$を
    $\displaystyle \alpha = \lim_{n \to \infty} n$
    と定める」という定義は不適切である.
    なぜならば,右辺は明らかに実数ではなく,
    この$\alpha$は「実数」とは呼べないからである.
    「実数$x$に対し,$y^2= x$を満たす実数$y$を対応させる関数を$f(x)$とする」
    という定義も不適切である.
    $x$が負であれば$y^2=x$を満たす実数$y$は存在せず,
    $x$が正であれば$y^2=x$を満たす実数$y$は2つ存在する.
    よって,この対応関係は「関数」と呼ぶことはできない.
    さらに,「ある点から近い点をその点の近傍と定める」
    という定義も許されない.
    与えられた点に対し,何が近傍で何が近傍でないかが
    はっきりとわからないからである.
    この場合は,「近い」とは
    どういうことなのかを明確に分かるように規定してやれば
    定義として正当なものになる.
  \end{ex}

  このように,すでに定義されたものを使って新しく定義をする場合,
  その定義がそれまでの議論と矛盾しないことを保証しなくてはならない.
  また,定義をする場合,その定義するものがなんであるかが
  一意に定まらなくてはならない
  \footnote{教育的配慮から,明確な意味で定義をするのを避け,
    直感的な議論に留めることも多い.
    例として,高校の教科書での極限の定義が挙げられる.
    本書では,「集合」や「属する」の定義や
    第\ref{chp:sequent}章で定義する用語のほとんどがそれに当てはまる.
  }
  .
  このことが保証されている場合,その定義は
  \index[widx]{well-defined@well-defined}
  \textbf{well-defined}であるという.
  また,定義がwell-definedでない場合,その定義は
  \index[widx]{ill-defined@ill-defined}
  \textbf{ill-defined}であるという.
  例\ref{ex:welldef1}で挙げた定義はともにill-definedである.

  \begin{ex}
    「実数$\gamma$を
    \begin{align*}
      \gamma = \lim_{n \to \infty} \left( \sum_{k=1}^{n} \frac{1}{k} - \log n \right)  
    \end{align*}
    と定める」という定義はwell-definedである.
    なぜならば,右辺の極限が確かに有限の実数であることが示せるからである.
  \end{ex}
  well-defined性に関する議論で身近なのは,代数学において同値類に演算を導入する場面であろう.
  \begin{ex}
    2以上の自然数$n$に対し,$n$を法とする剰余類
    全体の集合$\mathbb{Z} / n \mathbb{Z}$上に加法$+$を
    \begin{align*}
      [a] + [b] = [ a+b ] \quad \left( [a] , [b] \in \mathbb{Z} / n \mathbb{Z} \right)
    \end{align*}
    で定義する.このとき,この加法$+$はwell-definedである.
    なぜならば,よく知られているように,
    与えられた$[a] , [b] \in \mathbb{Z} / n \mathbb{Z}$に対して
    $[a+ b]$はその代表元のとり方によらず一意に定まることが示せるため,
    この加法は確かに「集合$\mathbb{Z}/n \mathbb{Z}$上に定義された加法」
    といえるからである.もしも結果が剰余類だけでなくその代表元に依存するならば,
    これは$\mathbb{Z} / n \mathbb{Z}$の2つの元を定めただけではその結果が確定しないことになる.
    それでは「$\mathbb{Z}/ n \mathbb{Z}$上に加法を定めた」とはいえない.
  \end{ex}

  このように,数学において何かを定義する場合,その
  well-defined性は必ず確かめなくてはならない.
  しかし,そのことは明らかであるか,
  明らかでなくとも容易であるとして省略されることも多い.

\section{数学的構造}
\label{sec:structure}
  19世紀以降,数学理論の考察の対象は,
  数や図形といった具体的なものから集合という根源的な
  ものへと移り変わり,大きく抽象化されていくこととなった.
  その原動力となったのが,
  CantorとDedekindによって創始された集合論と,
  \index[nidx]{Galois@Galois(ガロア)}
  Galois以来大きな発展を遂げた群論を中心とする代数学である.
  
  この数学の抽象化という流れと相性が良かったためか,
  Hilbertの提唱した形式主義という考え方が
  幾何学のみならず数学全体に広まることとなった.
  ところが,Hilbertの思い描いた「公理」と
  我々が現在思い描く「公理」には微妙に差異がある.
  典型的な例として,以下に挙げる群の公理が挙げられる.
  \begin{axiom}[群の公理] \label{axiom:group}
    空でない集合$G$と$G$の元$e,$そして$G$上の
    二項演算$\ast : G \times G \longrightarrow G$と単項演算${} ^{-1} : G \to G$
    が以下の条件を満たすとき,
    対$(G,e, \ast , {} ^{-1})$を群と呼び,$G$は演算$\ast$に関して$e$を単位元とする
    群をなすという:
    \begin{enumerate}[G1. ]
      \item $\forall x, y, z \in G (x \ast ( y \ast z)) ,$
      \item $ \forall x \in G (x \ast e = e \ast x = x),$
      \item $\forall x \in G (x \ast x^{-1} = x^{-1} \ast x = e).$
    \end{enumerate}
  \end{axiom}
  Hilbertが思い描いた「公理」というのは,
  あくまで確定した1つの対象を定式化するための命題である.
  しかし,群の公理系のモデルのなかで明らかに「異なる」と思われるものはいくつも発見できる.

  \begin{ex} \label{ex:groupmodel}
    (空でない)集合$X$に対し,$X$から自身への全単射全体の集合は,
    写像の合成に関して恒等写像を単位元とする群をなす.
    また,整数全体の集合$\mathbb{Z}$は,その加法に関して
    $0$を単位元とする群をなす.
    これらは「同じ」群であるが,両者は明らかに「異なる」と思われる.
    実際,前者は演算が可換ではないが,後者は演算が可換である.
  \end{ex}

  群の公理とは少し様子が異なる公理系を挙げておこう.

  \begin{ex} \label{ex:hilbertaxiom}
    実数論の公理系のモデルはどの2つも「本質的に同じ」である.
    すなわち,実数論の公理系のモデルは本質的にただ1つである
    \footnote{実はこのことは事実として
    正しいかどうかも含めてかなり込み入った事情がある.
    詳しく知りたい人は高階の論理やモデル理論について学ぶ必要がある.}
    .
  \end{ex}

  Hilbertは,「よい」公理系に対し,無矛盾性や独立性だけではなく,
  この「その公理系のモデルが本質的にただ1つである」という性質も要求していた.
  なぜそのような性質を要求したかは省略するが,
  Hilbertにとっては,群の公理やベクトル空間の公理といった
  同じ公理系のモデルで「異なる」モデルが存在するような公理系は
  「よい」公理系ではなかったことになる.

  公理系のモデルが「本質的に同じである」とき,そのモデルは
  \index[widx]{どうけい@同型 \, isomorphic}
  \emph{同型}(ismorphic)であるという.
  さらに,公理系のモデルがどの2つも同型である,すなわち
  その公理系のモデルが本質的にただ1つであるとき,
  その公理系は
  \index[widx]{はんちゅうせい@範疇性 \, categoricity}
  \emph{\ruby{範疇性}{はんちゅうせい}}(categoricity)
  をもつという.
  「本質的に同じである」というのがどういうことなのかということを
  述べるのはやや面倒なので省略するが,
  公理系に範疇性を要求していたHirbertにとっては,
  「公理系を考察する」ということは「ある特定の具体的な対象を考察する」
  ことと同じだったのである.

  ところが,Hilbertこのような態度は21世紀を生きる我々とは相容れないものがある.
  我々は公理系を考察するとき,「この公理系を満たす対象全体の共通点を探っていく」
  という態度をとっているはずである.
  これは,暗に「公理系を満たす対象は複数存在する」
  ということを認めているということを意味する.
  すなわち,公理を,ある特定の具体的な対象を定式化するためではなく,
  無数の対象に共通する性質を抽出して定式化するために用いているのである.
  
  このような「無数の対象に共通する性質を探っていく」という態度で
  数学理論を最初に構築したのは誰か,という疑問に答えるのは実に難しい.
  しかし,「無数の対象に共通する性質を探っていく」ことそのものに着目し,
  それを最初に定式化したのはおそらく
  \index[nidx]{Bourbaki@Bourbaki(ブルバキ)}
  Bourbakiであろう.

  フランスの数学者集団
  \footnote{Bourbakiは多数の若手数学者からなる集団であったが,
    当初はさも個人であるかのように活動していた.説明のしやすさを考え,
    ここでもBourbakiをさも個人であるかのように扱う.}
  Bourbakiは,19世紀のCauchyの時代以降標準的に用いられていた解析学の教科書に
  不満を持っていた.
  そこで,現代的な解析学の教科書を執筆しようということになったのだが,
  その作業があまりに膨大であったため,最終的には現代数学を
  厳密かつ公理論的に構築し直そうということになった.
  そうして1939年に『数学原論』という本の第1巻が出版されることとなる.

  「原論」とあるように,『数学原論』はただひたすらに一般から特殊へという
  流れで議論が進んでいく.
  しかし,Euclidの『原論』の時代とは当時得られていた数学の成果の量が段違いであったため,
  その量も『原論』とは比べ物にならない.
  日本語に翻訳されているものだけでも30巻はゆうに超え,
  今もなお続刊が執筆され続けている.
  Euclidの『原論』と同じように,『数学原論』で語られる事実の多くは
  Bourbakiが発見したものではなく,当時すでに得られていたものである.
  『数学原論』において画期的なのは,その考察対象を
  集合に対してある種の「性質」を付与するという方法で構成していったことである.

  集合に対してある種の「性質」を付与したものを
  \index[widx]{くうかん@空間 \, space}
  \emph{空間}(space)と呼び
  \footnote{この言い方だと群や環といった我々が素朴な意味で「空間」と呼んでいるものとは
    明らかに異なったものも「空間」ということになってしまうが,明示的に「空間」と
    呼ばれるのはベクトル空間や距離空間などの我々の幾何学的直感が
    通用するものがほとんどである.}
  ,
  付与するその「性質」のことを
  \index[widx]{すうがくてきこうぞう@数学的構造 \, mathmatical structure}
  \emph{数学的構造}(mathmatical structure)
  ,あるいは単に %
  \index[widx]{こうぞう@構造 \, structure|see{数学的構造}}
  \emph{構造}(structure)
  と呼ぶ.
  空間を考える文脈では,構造を与える前の集合をその空間の
  \index[widx]{しゅうごう@集合 \, set!だいしゅうごう@台--- \, underlying ---}
  \emph{台集合}(underlying set)という.
  そして,この数学的構造が満たすべき条件こそが
  現在でいうところの「公理」なのである.
  かなり曖昧な言い方であるが,明確な形で定義を述べるのは
  かなり面倒であるため省略させてもらうことにする.

  Bourbakiが導入した数学的構造の中で最も基本となるのは,
  代数構造・順序構造・位相構造の3つである.
  これらの構造を複数もっているとみなせる集合は少なくなく,
  たとえば実数全体の集合$\mathbb{R}$は上記3つの構造をすべてもっていると解釈できる.
  
  Bourbakiは「集合の上に構造を付与する」という思想の上で
  数学を構築していった.すなわち,構造を付与する前の集合は,
  何の意味ももたない単なる「モノ」の集まりということである.
  従って,集合を特徴づけるのは「モノ」の数のみということとなり,
  それさえ同じなら「集合としては」同じということになる.

  \begin{ex} \label{ex:setisomorphic}
    第\ref{chp:cardinal}章でみるように,
    自然数全体の集合$\mathbb{N}$と有理数全体の集合$\mathbb{Q}$
    の間には全単射が存在する.
    集合の濃度をその集合の個数と解釈するのならば,
    $\mathbb{N}$と$\mathbb{Q}$は「集合としては」まったく「同じ」ということになる.
  \end{ex}

  「$\mathbb{N}$と$\mathbb{Q}$がまったく同じ」と言われれば,
  ほとんどの人が違和感を抱くに違いない.
  しかしBourbaki流にいえば,その「違い」は$\mathbb{N}$と$\mathbb{Q}$を単なる集合
  とだけ考えたのでは決して発見できず,そこに付与された「構造」を見ることにより
  初めて発見できるということになる.
  実際,$\mathbb{N}$と$\mathbb{Q}$は(自然に持ち合わせているとみなせる)
  代数構造,順序構造がまったく異なる.
  代数構造が異なる根拠としては$\mathbb{Q}$は体であるが$\mathbb{N}$は
  体でないこと,順序構造が異なる根拠としては
  $\mathbb{N}$には最小元が存在するが$\mathbb{Q}$には最小元が存在しないこと
  などが挙げられる.

  「集合として」同じであっても,「構造」が違えばそれらの対象は
  異なるとみなされる.一方,
  「集合として」同じであり,かつその「構造」も同じであれば,
  それらの対象はたとえどのような形で記述されていたとしても「同じ」と
  みなされる.

  \begin{ex} \label{Rseidoukei}
    実数全体の集合$\mathbb{R}$と正の実数全体の集合$\mathbb{R}^+$は,
    $\mathbb{R}$についてはその加法,$\mathbb{R}^+$については
    その乗法の代数構造のみを考えると,
    これら2つは集合としても「同じ」で,代数構造も「同じ」である.
    実際,$\mathbb{R}$上で$\alpha , \beta$に対して$\alpha + \beta$
    という和を考えることは,$\mathbb{R}^+$上においては
    $e^{\alpha} \cdot e^{\beta}$という積を考えることに相当する.
  \end{ex}

  例\ref{Rseidoukei}が意味することは,加法群$( \mathbb{R} , 0, +,-)$と
  乗法群$( \mathbb{R}^+, 1, \cdot,{}^{-1}) $が「本質的に同じ」
  である,すなわちこの2つの群は群として同型であるということである.
  このことは通常$(\mathbb{R},0,+, -) \cong (\mathbb{R}^+ ,1,\cdot , {}^{-1})$
  と表記される.「群」という観点で考える限り,
  これら2つの対象は「同じ」とみなされる.
  これは,群論がこれら2つの対象に共通する性質のみを研究対象とすることを意味する.
  幾何学においては,エルランゲン・プログラムがちょうどそのような視点に沿って
  幾何学を構築すべきだと主張するものである.
  互いに移り合う変換群が存在するような2つの図形を同型と考えるのである.

  2つの対象が同型であることは,全単射であって,その写像と逆写像の両方が
  構造を保つものが存在することとして定式化される.
  同種の構造をもった2つの集合$A,B$において,
  全単射とは限らないが構造を保つような写像$f:A \longrightarrow B$が存在する場合,
  その写像$f$のことを
  \index[widx]{しゃぞう@写像 \, mapping!じゅんどうけいしゃぞう@準同型--- \, homomorphism}
  \emph{準同型写像}(homomorphism)という.
  準同型写像$f$が特に単射である場合,$A$と$f(A)$が同型である場合がある.
  このとき,$f$を
  \index[widx]{うめこみ@埋め込み \, embedding}
  \emph{埋め込み}(embedding)という.
  埋め込みが存在する場合,$A$と$f(A)$を同一視して,$A$が$B$の部分集合であるかのように
  扱うことができる.

  もちろん,群として同型であっても,他の構造に目を向ければ同型でないということはある.

  \begin{ex} \label{ex:Rseinotdoukei}
    $\mathbb{R}$には整列順序と呼ばれる順序構造$\leq '$を入れ,
    $\mathbb{R}^+$には通常の大小関係と同じ順序構造$\leq$を入れる.
    このとき,これら2つの順序集合は順序同型でない.
    実際,順序集合$(\mathbb{R}, \leq ')$は最小元をもつが,
    順序集合$( \mathbb{R}^+ ,\leq) $は最小元をもたない.
  \end{ex}

  このように,同型でない複数の対象が同じ公理系のモデルとして議論されているのが
  Bourbaki流の公理論的数学理論の最も大きな特徴と考えることができる.
  そのため,典型的な例とは明らかにかけ離れた対象も同じ土台に乗せられて
  議論されることがある.
  たとえば,$n$次元数ベクトル空間$\mathbb{R}^n$とある閉区間$[a,b]$上で定義された
  関数全体の集合$\mathbb{R} ^{[a,b]}$をベクトル空間とみなしたものが
  同じ(同型という意味ではない)
  ベクトル空間とみなされ,一緒くたにされて議論されていることには
  違和感を抱くこともあるだろう.
  もちろんこれら2つのベクトル空間は同型ではないため,
  「この2つは違う対象だろう」という直感は健全なものである.

  Bourbakiのこの「集合というまっさらな土台に構造を与え,
  それにより生まれる性質を研究する」という思想はすぐに数学全体に波及した.
  しかし,波及した学問分野は数学だけにとどまらなかった.
  フランスの文化人類学者
  \index[nidx]{Strauss@Strauss(ストロース)}
  Straussがオーストラリアの原住民の婚姻システムの「構造」を
  群論を用いて説明してみせたのである.
  このことを皮切りに,Bourbakiの思想は
  \index[widx]{こうぞう@構造 \, structure!しゅぎ@---主義 \, structuralism}
  \emph{構造主義}(structuralism)という名前で世界中に広まり,
  数学に限らない数多くの学問分野に取り入れられることとなった.

  現在では,数学的構造とその間の関係を研究する圏論や,
  数理論理学の手法を用いて数学的構造を研究するモデル理論といった
  数学的構造そのものが研究対象となっている理論も生み出され,
  そして発展している.
  残念ながら『数学原論』はこれらをカバーしてはいないが,
  構造主義という思想がこれらの理論に影響を与えていることは
  もはや疑う余地はないように思える.





  


  

  

  

% \section{よく使う数学界の方言}
% \label{sec:hougen}

