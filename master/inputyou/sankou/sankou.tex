\begin{thebibliography}{99}
 \bibitem{kigouronri} 前原 昭二『記号論理入門』,2005年,日本評論社 \\
 これほどわかりやすく書かれた記号論理の解説にはなかなかお目にかかれるものではない.
 数学を学ぶのであればとりあえず購入していい本である.
 とはいえ,内容はそれほど豊富というわけではない
 (それでも本格的に足を突っ込むのでなければ十分)ので,
 そのあたりは他の本に頼る必要がある.
 \bibitem{suurironri} 鹿島 亮『数理論理学(現代基礎数学)』,2009年,朝倉書店 \\
 本書のシークエント計算の解説に煮えきらなさを感じた方はこういう本を読むと良い.
 類書よりもたくさんの話題が取り上げられているお買い得パックである.
 \bibitem{kada} 嘉田 勝『論理と集合から始める数学の基礎』,2008年,日本評論社 \\
 ページ数が少ない割にたくさんの話題が解説されている.
 本書よりも情報科学に必要な話題が多く取り上げられており,
 情報系の学生には自信を持っておすすめできる1冊である.
 \bibitem{tukuru} 戸田山 和久『論理学をつくる』,2000年,名古屋大学出版会 \\
 本当の意味で0からの入門書が欲しい人はこれ.
 fitch styleと呼ばれる流儀で演繹を行っている.
 とにかく分厚い.本書の解説がいかに薄っぺらなものかがよくわかる本である.
 \bibitem{yosida} 戸次 大介『数理論理学』,2012年,東京大学出版会 \\
 本書のシークエント計算の体系はこの本がもとになっている.
 本書の読者にはもっとも勧めたい一冊であるが,
 演習問題の解答がなく,独習には少々つらいかもしれない.
 \bibitem{kunen} Kenneth Kunen(藤田 博司 訳)『集合論-独立性証明への案内』,2008年,日本評論社 \\
 公理的集合論に関する本.研究分野として集合論を学ぼうとするような人間が読む本である.
 とはいえ,そんな人間なら邦訳ではなく原著を読むべきである.
 一般人は図書館あたりでちょこっとチラ見する程度でいいかも.
\bibitem{utidaset} 内田 伏一『集合と位相(数学シリーズ)』,1986年,裳華房 \\
 集合論の入門書としては定番の1冊.
 本書の次に読むならこのあたりを読むと良い.
 \bibitem{utidatopo} 内田 伏一『位相入門』,1997年,裳華房 \\
 同著者による本.{\cite{utidaset}}を
 大学の講義で使いやすいよう書き直したものらしい.
 位相空間論にさっさとたどり着きたいなら{\cite{utidaset}}よりも
 こちらを手に取るべきかもしれない.
 ただ1点残念なのは論理記号の使い方がよろしくないことである.
 \bibitem{matsuzaka} 松坂 和夫『集合・位相入門』,1968年,岩波書店 \\
 とてもていねいな記述がなされており読みやすい.
 おそらく高校生でも問題なく読める.
 最近新装版が出たらしい.
 \bibitem{30kouset} 志賀 浩二『集合への30講(数学30講シリーズ)』,1988年,朝倉書店 \\
 有名な30講シリーズの中の1冊.わかりやすいと評判らしい.
 \bibitem{30koutopo} 志賀 浩二『位相への30講(数学30講シリーズ)』,1988年,朝倉書店 \\
 \cite{30kouset}の続刊とも言うべき本.基本的に集合論と位相空間論はセットである.
 \bibitem{naze} 藤田 博司『「集合と位相」をなぜ学ぶのか---数学の基礎として根づくまでの歴史』,2018年,技術評論社 \\
 この本は,集合論や位相空間論
 がなぜ数学の基礎として
 認識されるようになったのかを
 解説した本である.
 健全な学習のためにも,
 こういう教科書ではない「読み物」
 もたくさん読んでおくべきである.
  \bibitem{latex} 奥村 晴彦・黒木 裕介『{\LaTeXe}美文書作成入門』,2017年,技術評論社 \\
 この本は数学の専門書ではなく,数学を学ぶのなら絶対に知らなければならない
 {\LaTeX}というフリーの組版システムの解説書である.
 まさかいつまでも手書きで文章を書くわけでもあるまいから,
 ちょっとくらい手を出してもバチは当たらないだろうと思う.
 \bibitem{bunsyo} 結城 浩『数学文章作法 基礎編』,2013年,筑摩書房 \\
 この本は数学における文章の書き方に焦点を絞った本である.
 豊富な具体例をもとにして,どう文章を書いたらよいのかが丁寧に解説されている.
 「基礎編」とあるように,書いてある内容はごくごく当たり前のことだが,
 それゆえ忘れがちである.
 具体例が豊富であるため,その内容がどう文章に生かされているのかが
 実感をもってわかることであろう.
\end{thebibliography}

    
