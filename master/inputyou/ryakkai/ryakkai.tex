\chapter{演習問題解答例} \label{answer}
\begin{description}
  \item[\refque{chp:sequent.sec:ronri.que:singihantei}] \mbox{}
  \begin{enumerate}
   \item 偽
   \item 真
   \item 真
   \item 偽(反例は$x=3,  y=2$など)
   \item 偽
   \item 真
  \end{enumerate}
\item[\refque{chp:sequent.sec:hensuu.que:sokubakusingi}] \mbox{} 
  \begin{enumerate}
   \item 偽(反例は$x=1$など)
   \item 真
   \item 偽
   \item 真
   \item 真 ($x=3$とでもすればよい)
   \item 偽
  \end{enumerate}
\item[\refque{chp:sequent.sec:hituyoujubun.que:xhituyoujubun}] \mbox{} 
  \begin{enumerate}
   \item 正しい
   \item 正しい
   \item 正しい
  \end{enumerate}
\item[\refque{que:kigoukaranihongo}] \mbox{} \\
  ここに挙げるのはあくまで一例である.
  細かい言い回しを考えれば,解答として適切なものはいくつも考えられるであろう.
  \begin{enumerate}
    \item  $F(x)$となる$x$がただ1つ存在する.
    \item 任意の$\varepsilon >0$に対して$\delta >0$が存在して,
      $0< \lvert x- a \rvert < \delta$を満たす任意の$x \in I$に対して
      $\lvert f(x) -A \rvert < \varepsilon$が成り立つ.
    \item 任意の$y \in Y$に対して$x \in X$が存在して,$y=f(x)$となる.
    \item $f(x_1 )=f(x_2) $を満たす任意の$x_1, x_2 \in X$に対して$x_1 = x_2$が成り立つ.
    \item 任意の$a,b >0$に対して自然数$N$で$Na >b$となるものがとれる.
  \end{enumerate}

\item[\refque{que:nihongokarakigou}] \mbox{} \\
  \begin{enumerate}
    \item $\forall x ( x \in A \to x \in B).$
      (「必ず」とあることを考えれば$x$には全称記号をつけるのが妥当であろう)
    \item $\forall \varepsilon >0 \exists N \in \mathbb{N} \forall n,m \in \mathbb{N}
      (n,m \geq N \to \lvert a_n - a_m \rvert < \varepsilon ).$
    \item $\forall x,y \in \mathbb{R} (x^2+y^2=1 \to \exists \theta \in \mathbb{R} 
      ( 0 \leq \theta <2 \pi \land (x= \cos \theta \land y= \sin \theta ))).$
    \item $\exists c \in \mathbb{R}(\forall x \in S ( x \leq c ) \land 
      \forall M \in \mathbb{R} (\forall x \in S (x \leq M) \to c \leq M)).$
  \end{enumerate}
\item[\refque{que:haityurituouyou}] \mbox{} \\  
  $\sqrt{2}^{\sqrt{2}}$が有理数の場合には$a = b= \sqrt{2}$とすればよい.
  $\sqrt{2}^{\sqrt{2}}$が有理数でない場合には,
  $\sqrt{2}^{\sqrt{2}}$は無理数であるから
  $a = \sqrt{2} ^{\sqrt{2}},  b= \sqrt{2}$とすればよい.

\item[\refque{que:meidaiketugouritu}] \mbox{} \\
  式\eqref{eq:bekiland}を示す.
  まずはシークエント$A \land A \Longrightarrow A$
  を導出する.
  \begin{enumerate}[1. ]
    \item $A \land A \Longrightarrow A \land A$ \quad [始式]
    \item $A \land A \Longrightarrow A$ \quad 
      [1.から$\land$除去による]
  \end{enumerate}
  次に,シークエント$A \Longrightarrow A \land A$を導出する.
  \begin{enumerate}[1. ]
    \item $A \Longrightarrow A$ \quad [始式]
    \item $A \Longrightarrow A$ \quad [始式]
    \item $A \Longrightarrow A \land A$ \quad 
      [$1.,2.$から$\land$導入による]
  \end{enumerate}
  式\eqref{eq:bekilor}を示そう.
  まずはシークエント$A \lor A \Longrightarrow A$
  を導出する.
  \begin{enumerate}[1. ]
    \item $A \lor A \Longrightarrow A$ 
      \quad [始式]
    \item $A \Longrightarrow A$ \quad [始式]
    \item $A \Longrightarrow A$ \quad [始式]
    \item $A \lor A \Longrightarrow A$ \quad [$1.,2.,3.$から
      $\lor$除去による]
  \end{enumerate}
  次に,シークエント$A \Longrightarrow A \lor A$を導出する.
  \begin{enumerate}[1. ]
    \item $A \Longrightarrow A$ \quad [始式]
    \item $A \Longrightarrow A \lor A$ \quad [1.から$\lor$導入による]
  \end{enumerate}

  あとはシークエント$A \lor ( B \lor C) \Longrightarrow 
  (A \lor B ) \lor C$の導出を与え,残りは省略する.

  \begin{enumerate}[1. ]
    \item $A \lor ( B \lor C) \Longrightarrow A \lor (B \lor C)$
           \quad [始式]
    \item $A \Longrightarrow A$ \quad [始式]
    \item $A \Longrightarrow A \lor B$ \quad [2.から$\lor$導入による]
    \item $A \Longrightarrow (A \lor B ) \lor C$ \quad [3.から$\lor$導入による]
    \item $B \lor C \Longrightarrow B \lor C$ \quad [始式]
    \item $B \Longrightarrow B$ \quad [始式]
    \item $B \Longrightarrow A \lor B$ \quad [6.から$\lor$導入による]
    \item $B \Longrightarrow ( A \lor B) \lor C$ \quad [7.から$\lor$導入による]
    \item $C \Longrightarrow C$ \quad [始式]
    \item $C \Longrightarrow (A \lor B ) \lor C$ \quad [9.から$\lor$導入による]
    \item $B \lor C \Longrightarrow (A \lor B ) \lor C$
           \quad [$5., 8., 10.$から$\lor$除去による]
    \item $A \lor ( B \lor C) \Longrightarrow (A \lor B ) \lor C$
           \quad [$1., 4., 11.$から$\lor$除去による]
  \end{enumerate}

\item[\refque{que:Dnarabanot}] \mbox{} \\
  式\eqref{eq:Dnotall}を示す.
  \begin{align*}
    \lnot \forall x \in D ( F(x) ) & \equiv \lnot \forall x ( x \in D \to F(x) ) \\
                                   & \equiv \exists x ( x \in D \land \lnot F(x) ) \\
                                   & \equiv \exists x \in D ( \lnot F(x) ) .
  \end{align*}
  次に式\eqref{eq:Dnotexists}を示す.
  \begin{align*}
    \lnot \exists x \in D(F(x)) & \equiv \lnot \exists x ( x \in D \land F(x) ) \\
                                & \equiv \forall x \lnot ( x \in D \land F(x) ) \\
                                & \equiv \forall x ( \lnot ( x \in D) \lor \lnot F(x) ) \\
                                & \equiv \forall x ( x \in D \to \lnot F(x) ) \\
                                & \equiv \forall x \in D ( \lnot F(x) ).
  \end{align*}
\item[\refque{que:tototo}] \mbox{} \\
  シークエント$A \to ( B \to C) \Longrightarrow A \land B \to C$の導出をする.
    \begin{enumerate}[1. ]
      \item $A \to ( B \to C) \Longrightarrow A \to (B \to C)$ 
             \quad [始式]
      \item $A \land B \Longrightarrow A \land B$ \quad [始式]
      \item $A \land B \Longrightarrow A$ \quad [2.から$\land$除去による]
      \item $A \land B ,  A \to ( B \to C) \Longrightarrow B \to C$
             \quad [$1., 3.$から$\to$除去による]
      \item $A \land B \Longrightarrow B$ \quad [2.から$\land$除去による]
      \item $A \land B ,  A \to ( B \to C) \Longrightarrow C$
             \quad [$4., 5.$から$\to$除去による]
      \item $A \to (B \to C ) \Longrightarrow A \land B \to C$
             \quad [6.から$\to$導入による]
    \end{enumerate}
    逆向きのシークエントも同様に導出できる.
\item[\refque{que:Peirce}] \mbox{}
  \begin{enumerate}[1. ]
    \item $(A \to B)  \to A \Longrightarrow (A \to B) \to A$ \quad [始式] 
    \item $\lnot A \Longrightarrow \lnot A$ \quad [始式(背理法で示す)]
    \item $A \Longrightarrow A$ \quad [始式]
    \item $\lnot A ,  A \Longrightarrow \curlywedge$ \quad [$2., 3.$から$\lnot$除去による]
    \item $\lnot A ,  A \Longrightarrow B$ \quad [4.から矛盾による]
    \item $\lnot A \Longrightarrow A \to B$ \quad [5.から$\to$導入による]
    \item $\lnot A,  (A \to B) \to A \Longrightarrow A$ \quad [$1., 6.$から$\to$除去による] 
    \item $\lnot A,  (A \to B) \to A \Longrightarrow \curlywedge$
           \quad [$2., 7.$から$\lnot$除去による]
    \item $(A \to B) \to A \Longrightarrow \lnot \lnot A$ \quad [8.から$\lnot$導入による]
    \item $(A \to B) \to A \Longrightarrow A$ \quad [9.から2重否定の除去による]
  \end{enumerate}
\item[\refque{que:togentei}] \mbox{} \\
  定理\ref{thm:genteito}と同様に示せる.
\item[\refque{que:genteilandor}] \mbox{} \\
  シークエント$\forall x (F(x) \lor A) \Longrightarrow \forall x F(x) \lor A$
  の導出を与えておく.残りのものは容易である.
  \begin{enumerate}[1. ]
    \item $\forall x (F(x) \lor A) \Longrightarrow 
      \forall x(F(x) \lor A) $ \quad [始式]
    \item $\lnot ( \forall x F(x) \lor A ) 
      \Longrightarrow \lnot ( \forall x F(x) \lor A)$
      \quad [始式(背理法で示す)]
    \item $\forall x (F(x) \lor A) \Longrightarrow 
      F(a) \lor A $ \quad [1.から$\forall$除去による
      ($a$は新たな自由変数)]
    \item $F(a) \Longrightarrow F(a)$ \quad [始式]
    \item $A \Longrightarrow A$ \quad [始式]
    \item $\lnot A \Longrightarrow \lnot A$ \quad [始式]
    \item $ A , \lnot A \Longrightarrow \curlywedge$
      \quad [$5.,6.$から$\lnot$除去による]
    \item $A , \lnot A \Longrightarrow F(a)$
      \quad [7.から矛盾による]
    \item $\lnot A , \forall x ( F(x) \lor A)
      \Longrightarrow F(a)$ \quad 
      [$3.,4.,8.$から$\lor$除去による]
    \item $\lnot A , \forall x (F(x) \lor A)
      \Longrightarrow \forall x F(x)$
      \quad [9.から$\forall$導入による]
    \item $\lnot A , \forall x (F(x) \lor A) 
      \Longrightarrow \forall x F(x) \lor A$
      \quad [10.から$\lor$導入による]
    \item $\lnot (\forall x F(x) \lor A) , \lnot A , 
      \forall x (F(x) \lor A) \Longrightarrow \curlywedge$
      \quad [$2.,11.$から$\lnot$除去による]
    \item $\lnot ( \forall x F(x) \lor A),
      \forall x( F(x) \lor A) \Longrightarrow \lnot \lnot A$
      \quad [12.から$\lnot$導入による]
    \item $\lnot ( \forall x F(x) \lor A) , 
      \forall x (F(x) \lor A) \Longrightarrow A$
      \quad [13.から2重否定の除去による]
    \item $\lnot ( \forall x F(x) \lor A ),
      \forall x (F(x) \lor A) \Longrightarrow \forall x F(x) \lor A$
      \quad [14.から$\lor$導入による]
    \item $\lnot ( \forall x F(x) \lor A ) ,
      \forall x (F(x) \lor A) \Longrightarrow \curlywedge$
      \quad [$2.,15.$から$\lnot$除去による]
    \item $\forall x (F(x) \lor A ) \Longrightarrow 
      \lnot \lnot ( \forall x F(x) \lor A)$
      \quad [16.から$\lnot$導入による]
    \item $\forall x (F(x) \lor A) \Longrightarrow 
      \forall x F(x) \lor A$
      \quad [17.から2重否定の除去による]
  \end{enumerate}
\item[\refque{que:genteikoukan}] \mbox{} \\
  シークエント$\exists x \exists y F(x,y) \Longrightarrow \exists y \exists x F(x,y)$
  を導出しよう.
  \begin{enumerate}[1. ]
    \item $\exists x \exists y F(x,y) \Longrightarrow \exists x \exists y F(x,y)$
           \quad [始式]
    \item $\exists y F(a, y) \Longrightarrow \exists y F(a,y)$
           \quad [始式($a$は新たな自由変数)]
    \item $F(a, b) \Longrightarrow F(a,b)$ \quad [始式($b$は新たな自由変数)]
    \item $F(a,b) \Longrightarrow \exists x F(x,b)$ \quad [3.から$\exists$導入による]
    \item $F(a,b) \Longrightarrow \exists y \exists x F(x,y)$
           \quad [4.から$\exists$導入による]
    \item $\exists y F(a,y) \Longrightarrow \exists y \exists x F(x,y)$
           \quad [$2., 5.$から$\exists$除去による]
         \item $\exists x \exists y F(x,y) \Longrightarrow \exists y \exists x F(x,y)$
           \quad [$1., 6.$から$\exists$除去による]
  \end{enumerate}
  逆向きのシークエントもまったく同様にして導出できることは明らかであろう.
\item[\refque{que:circdouti}] \mbox{} \\
  cut規則を繰り返し用いればよい.
\item[\refque{que:taisyousuii}] \mbox \\
  式\eqref{eq:taisyouritu}を導出する.
  \begin{enumerate}[1. ]
    \item $a =b \Longrightarrow a =a \to b =a$ \quad [置換法則]
    \item $\qquad \Longrightarrow a =a$ \quad [反射律]
    \item $a = b \Longrightarrow b =a$ \quad [$1., 2.$から$\to$除去による]
  \end{enumerate}
  次に,式\eqref{eq:suiiritu}を導出する.
  \begin{enumerate}[1. ]
    \item $a =b \land b=c \Longrightarrow a = b \land b =c$ \quad [始式]
    \item $b=a \Longrightarrow b=c \to a=c$ \quad [置換法則]
    \item $a = b \Longrightarrow b =a$ \quad [対称律]
    \item $a=b \land b=c \Longrightarrow a=b$ \quad [1.から$\land$除去による]
    \item $a=b \land b=c \Longrightarrow b=a$ \quad [$3., 4.$からcutによる]
    \item $a=b \land b=c \Longrightarrow b=c \to a=c$ \quad [$2., 5.$からcutによる]
    \item $a =b \land b=c \Longrightarrow b=c$ \quad [1.から$\land$除去による]
    \item $a=b \land b=c \Longrightarrow a=c$ \quad [$6., 7.$から$\to$除去による]
  \end{enumerate}
\item[\refque{que:2tusonzaihitei}] \mbox{} \\
  証明済みの関係式を用いれば容易であろう.
\item[\refque{que:tadahitotuexists}] \mbox{} \\
  式\eqref{eq:uniquetakadaka}の導出をする.他は省略する.
  \begin{enumerate}[1. ]
    \item $\exists x \forall y (F(y) \to y=x)
           \Longrightarrow \exists x \forall y (F(y) \to y=x)$ \quad [始式]
    \item $F(a) \land F(b) \Longrightarrow F(a) \land F(b) $ 
           \quad [始式($a,  b$は新たな自由変数)]
    \item $\forall y (F(y) \to y=c) \Longrightarrow \forall y (F(y) \to y=c)$
           \quad [始式($c$は新たな自由変数)]
    \item $F(a) \land F(b) \Longrightarrow F(a)$ \quad [2.から$\land$除去による]
    \item $\forall y (F(y ) \to y=c ) \Longrightarrow F(a) \to a =c $
           \quad [3.から$\forall$除去による]
    \item $\forall y (F(y) \to y=c) ,  F(a) \land F(b) \Longrightarrow a =c$
           \quad [$3., 4.$から$\to$除去による]
    \item $F(a) \land F(b) \Longrightarrow F(b)$ \quad [2.から$\land$除去による]
    \item $\forall y (F(y) \to y=c) \Longrightarrow F(b) \to b=c$
           \quad [3.から$\forall$除去による]
    \item $\forall y (F(y) \to y =c) ,  F(a) \land F(b) \Longrightarrow b =c$
           \quad [$7., 8.$から$\to$除去による]
    \item $b=c \Longrightarrow c =b$ \quad [対称律]
    \item $\forall y (F(y) \to y =c) ,  F(a) \land F(b) 
           \Longrightarrow c =b$ \quad [$9., 10.$からcutによる]
    \item $\forall y (F(y) \to y=c) \Longrightarrow a =c \land c=b$ 
           \quad [$6., 11.$から$\land$導入による]
    \item $a=c \land c=b \Longrightarrow a=b$ \quad [推移律]
    \item $\forall y (F(y) \to y=c) ,  F(a) \land F(b) 
           \Longrightarrow a =b$ \quad [$12., 13.$からcutよる]
    \item $\forall y (F(y) \to y =c) \Longrightarrow 
           F(a) \land F(b) \to a =b$ \quad [14.から$\to$導入による]
    \item $\forall y (F(y) \to y=c) \Longrightarrow 
           \forall y (F(a) \land F(y) \to a =y)$ \quad [15.から$\forall$導入による]
    \item $\forall y(F(y) \to y=c) \Longrightarrow 
           \forall x \forall y (F(x) \land F(y) \to x=y)$ \quad [16.から$\forall$導入による]
    \item $\exists x \forall y (F(y) \to y=x) \Longrightarrow 
           \forall x \forall y (F(x) \land F(y) \to x=y )$
           \quad [$1., 17.$から$\exists$除去による]
  \end{enumerate}




\item[\refque{que:varnothing}] \mbox{} \\
  集合$A$を任意にとり,$\varnothing \subset A$を示す.
  任意の$x$に対し,空集合の定義から
  $x \notin \varnothing$すなわち$\lnot ( x \in \varnothing)$
  が成り立つので$x \in \varnothing \to x \in A$が成り立つ.
  従って,$\forall x( x \in \varnothing \to x \in A)$が成り立つので
  $\varnothing \subset A$が成り立つ.

  次に,$A \subset \varnothing$を満たす集合$A$を任意にとると,
  $\forall x (x \in A \to x \in \varnothing)$が成り立つ.
  空集合の定義から任意の$x$に対して$\lnot (x \in \varnothing)$が成り立つので,
  $\forall x (x \in A \to x \in \varnothing)$が成り立つのは$A$が
  どんな$x$に対しても$x \notin A$を満たすとき,
  すなわち$A = \varnothing$である場合であり,またそのときに限る.
\item[\refque{que:unionintersection}] \mbox{} \\
  $
    A \cup B = \set{1,  3,  5,  6,  7,  9,  12},   
    A \cap B = \Set{3,  9} .
  $
\item[\refque{que:naihouequal}] \mbox{} \\
  $A,  B$を外延的記法で表すと,
  どちらも同じ集合$\set{ 3,  5,  7}$
  を表していることがわかる.従って$A=B$.
\item[\refque{que:univarseset}] \mbox{} \\
  式\eqref{eq:ccA}を示す.任意の$x \in U$に対し,
  \begin{align*}
    x \in (A^c)^c & \equiv x \notin A^c \\
                  & \equiv \lnot (x \in A ^c) \\
                  & \equiv \lnot ( x \notin A ) \\
                  & \equiv \lnot \lnot (x \in A) \\
                  & \equiv x \in A . 
  \end{align*}
  よって,$\forall x \in U( x \in (A ^c)^c \rightleftarrows x \in A)$
  が成り立つので$(A^c)^c=A$となる.

  次に,式\eqref{eq:AAcU}を示す.
  $x \in U$を任意にとると,排中律から$x \in A \lor \lnot ( x \in A)$が成り立つ.
  よって,$x \in A \lor x \notin A$,すなわち$x \in A \lor x \in A^c$であり,
  これより$x \in A \cup A^c$を得る.
  従って,$\forall x( x \in U \to x \in A \cup A^c)$が成り立つので
  $U \subset A \cup A^c $である.
  また,$A \subset U,  A^c \subset U$より
  $A \cup A^c \subset U$だから$A \cup A^c =U$となる.

  式\eqref{eq:AAcvarnothing}を示す.
  $A \cap A^c \neq \varnothing$と仮定すると,
  $x \in A \cap A^c$となる$x$が存在する.
  しかし,この$x$は$x \in A \land x \in A^c$すなわち$x \in A \land \lnot(x \in A)$
  を満たさなくてはならず矛盾する.
  よって$A \cap A^c = \varnothing$である.
  
  式\eqref{eq:Ucvarnothing}と式\eqref{eq:varnothingcU}の導出は
  各自試みられたい.

  最後に式\eqref{eq:settaiguu}を示す.
  \begin{align*}
    A \subset B & \equiv \forall x \in U ( x \in A \to x \in B) \\
                & \equiv \forall x \in U ( \lnot (x \in B) \to \lnot (x \in A)) \\
                & \equiv \forall x \in U ( x \notin B \to x \notin A) \\
                & \equiv \forall x \in U ( x \in B^c \to x \in A^c) \\
                & \equiv B^c \subset A^c.
  \end{align*}
\item[\refque{que:mapsubset}] \mbox{} \\
  式\eqref{eq:AfinfA}と式\eqref{eq:ffinBB}は一般に成り立つ.
  証明は容易であろう.
  しかし,式\eqref{eq:finfAA}と式\eqref{eq:BffinB}は一般に成り立つとは限らない.
  読者は反例の構成を試みよ.
\item[\refque{que:ZNmapex}] \mbox{} \\
  \begin{enumerate}[(1) ]
    \item 写像$f: \mathbb{Z} \longrightarrow \mathbb{N}$を
      \begin{align*}
        f(n) = 1 \quad ( n \in \mathbb{Z})
      \end{align*}
      と定めれば,明らかに$f$は単射でも全射でもない.
    \item 写像$f: \mathbb{Z} \longrightarrow \mathbb{N}$を
      \begin{align*}
        f(n) = \left\{
        \begin{aligned} 
          4n-1 \quad & ( \text{$n$が正の整数のとき} ) , \\
          -4n+1 \quad & (\text{それ以外のとき} ) 
        \end{aligned}
        \right.
      \end{align*}
      と定めると,$f$は単射であるが全射でない.
    \item 写像$f: \mathbb{Z} \longrightarrow \mathbb{N}$を
      \begin{align*}
        f(n) = \left\{
          \begin{aligned}
            n \quad & ( \text{$n$が正の整数のとき} ) , \\
            1 \quad & ( \text{それ以外のとき} )
          \end{aligned}
        \right.
      \end{align*}
      と定めると,$f$は全射であるが単射でない.
    \item 写像$f: \mathbb{Z} \longrightarrow \mathbb{N}$を
      \begin{align*}
        f(n) = \left\{
          \begin{aligned}
            2n - 1 \quad & ( \text{$n$が正の整数のとき} ) , \\
            -2n  \quad \; \; &  ( \text{それ以外のとき} )
          \end{aligned}
        \right.
      \end{align*}
      と定めると,$f$は全単射である.
      読者はこのことを確かめよ.
  \end{enumerate}
\item[\refque{que:injecsurjec}] \mbox{} \\
  $X$が空でないことから,$x_0 \in X$がとれる.
  $f$が単射であることから,任意の$y \in f(X)$に対し,
  $y=f(x)$を満たす$x \in X$がただ1つ存在する.
  よって,$y \in f(X)$に対して$y=f(x)$を満たす$x \in X$
  を対応させる写像$h: f(X) \longrightarrow X$
  を定義でき,しかも$h$は全単射である.
  そこで,写像$g: Y \longrightarrow X$を
  \begin{align*}
    g(y) = \left\{
      \begin{aligned}
          h (y) \quad & ( \text{$y \in f(X)$のとき} ) , \\
          x_0 \quad \ & ( \text{それ以外のとき} ) 
      \end{aligned}
      \right.
  \end{align*}
  と定めれば,$g$は$Y$から$X$への全射である.

\item[\refque{que:injecsurjecsubset}] \mbox{} \\
  問\ref{que:mapsubset}も参照せよ.
  \begin{enumerate}[(1) ]
    \item $f$が単射である場合には$A= f^{-1}(f(A))$が成り立つ.
      $A \subset f^{-1}(f(A))$はつねに成り立つので
      $f^{-1}(f(A)) \subset A$が成り立つことを示せばよい.
    \item $f$が全射である場合には$B = f(f^{-1}(B))$が成り立つ.
      $f(f^{-1}(B)) \subset B$はつねに成り立つので
      $B \subset f(f^{-1}(B))$が成り立つことを示せばよい.
   \end{enumerate}
 
 \item[\refque{que:emptymapping}] \mbox{} \\
  空写像$f: \varnothing \longrightarrow A$は
  $\forall x_1 , x_2 \in \varnothing ( f(x_1 ) = f(x_2) \to x_1 = x_2)$
  すなわち$\forall x_1 , x_2 ( x_1 , x_2 \in \varnothing \to ( f(x_1)=f(x_2) \to x_1 = x_2))$
  を満たすから単射である.また,$f$はつねに単射であるから,
  $f$が全単射になるのは$f$が全射でもあるときである.$f$が全射となるのは
  $\forall y \in A \exists x \in \varnothing ( y = f(x))$
  すなわち$\forall y \exists x (y \in A \to x \in \varnothing \land y=f(x))$
  が成り立つときであり,これが成り立つのは$A$が空の場合であり,
  またそのときのみである.


\item[\refque{que:mapassociative}] \mbox{} \\
  $h \circ ( g \circ f)$と$(h \circ g ) \circ f$はともに
  $X$から$W$への写像である.
  従って,任意の$x \in X$に対して
  $(h \circ ( g \circ f))(x) = ((h \circ g) \circ f) (x)$
  が成り立つことを示せばよい.
  これは,両辺がどちらも$h(g(f(x)))$と等しいことから明らかであろう.
\item[\refque{que:mapinjesurjecomp}] \mbox{} \\
  (1)を示す.$f(x_1)=f(x_2)$となる$x_1 ,  x_2 \in X$を任意にとる.
  このとき,$g(f(x_1))=g(f(x_2))$すなわち
  $(g \circ f)(x_1) = (g \circ f)(x_2)$が成り立つ.
  $g \circ f$が単射であることから$x_1 =x_2$となるので,
  $f$は単射である.

  (2)を示そう.$z \in Z$を任意にとると,$g \circ f$が全射であることから
  $(g \circ f)(x)=z$となる$x \in X$が存在する.
  このとき,$y = f(x) \in Y$とおくと,
  $g(y) = g(f(x)) = (g \circ f)(x) =z$となる.
  従って$g$は全射である.
\item[\refque{que:compinvset}] \mbox{} \\
  まずは式\eqref{que:compinvset}の両辺が集合であることを認識しなければならない.
  あとは
  $\forall x \in X(
  x \in (g \circ f)^{-1}(C) \rightleftarrows x \in f^{-1}(g^{-1}(C)))$
  が成り立つことを示すだけである.これは逆像と合成写像の定義から明らかである.
\item[\refque{que:invcomp}] \mbox{} \\
  式\eqref{eq:invcompX}を示す.
  $f^{-1} \circ f$と$1_X$はともに$X$から自身への写像であるから,
  任意の$x \in X$に対して$(f^{-1} \circ f)(x) = 1_X(x) $
  が成り立つことを示せばよい.
  $1_X(x)=x$だから,結局$(f^{-1} \circ f)(x) = x$を示せばよい.
  逆写像の定義から,$(f^{-1} \circ f)(x) = f^{-1}(f(x)) = x$
  となる.
  式\eqref{eq:invcompY}も同様である.
\item[\refque{que:mapbijeide}] \mbox{} \\
  必要性は問\ref{que:invcomp}の結果より明らか.十分性に関しては,
  問\ref{que:mapinjesurjecomp}の結果と恒等写像が全単射であることを用いれば,
  $f$が全単射であることがわかり,$f$の逆写像$f^{-1}$が定義できることがいえる.
  あとは$g=f^{-1}$を示すだけであるが,
  問\ref{que:mapassociative}の結果を用いれば
  \begin{align*}
    g & = g \circ 1_Y \\
      & = g \circ ( f \circ f^{-1} ) \\
      & = ( g \circ f) \circ f^{-1} \\
      & = 1_X \circ f^{-1} \\
      & = f^{-1}.
  \end{align*}
\item[\refque{que:invcompgf}] \mbox{} \\
  問\ref{que:invcomp}と問\ref{que:mapbijeide}の結果を用いれば容易.
\item[\refque{que:taisyousa}] \mbox{} \\
  式\eqref{eq:symdiftaisyou}と式\eqref{eq:symdiftanigen}は明らかであろう.
  残りを示すことにする.まずは式\eqref{eq:symdifcupcap}を示す.
  \begin{align*}
    (A \cup B) - (A \cap B) & = (A \cup B) \cap (A \cap B)^c \\
                            & = (A \cup B) \cap (A^c \cup B^c) \\
                            & = (A \cap (A^c \cup B^c)) \cup (B \cap (A^c \cup B^c)) \\
                            & = (A \cap A^c) \cup (A \cap B^c) \\
                            & \hspace{1cm} \cup (B \cap A^c) \cup (B \cap B^c) \\
                            & = \varnothing \cup (A \cap B^c) \cup (B \cap A^c) \cup \varnothing \\
                            & = (A -B) \cup (B-A) \\
                            & = A \bigtriangleup B.
  \end{align*}
  次に,式\eqref{eq:symdifketugou}を示す.集合$A,  B$に対して
  \begin{align*}
    (A \bigtriangleup B)^c 
    & = ((A \cup B) - (A \cap B) )^c \\
    & = ((A \cup B) \cap (A \cap B)^c)^c \\
    & = (A \cup B)^c \cup (A \cap B) \\
    & = (A^c \cap B^c) \cup (A \cap B)
  \end{align*}
  となることに注意して,
  \begin{align*}
    (A \bigtriangleup B) \bigtriangleup C 
    & = (((A \bigtriangleup B)-C)) \cup (C- (A \bigtriangleup B)) \\
    & = (((A-B) \cup (B-A)) \cap C^c) \\
    & \hspace{1cm} \cup (C \cap (A \bigtriangleup B)^c) \\
    & = (((A \cap B^c) \cup (B \cap A^c)) \cap C^c) \\
    & \hspace{1cm} \cup ((A^c \cap B^c) \cup (A \cap B) \cap C) \\
    & = (A \cap B^c \cap C^c ) \cup (B \cap A ^c \cap C^c) \\
    & \hspace{1cm} \cup (A^c \cap B^c \cap C) \cup (A \cap B \cap C) \\
    & = (A \cap B^c \cap C^c ) \cup (A^c \cap B \cap C^c) \\
    & \hspace{1cm} \cup (A^c \cap B^c \cap C) \cup (A \cap B \cap C).
  \end{align*}
  従って,
  \begin{align*}
    A \bigtriangleup (B \bigtriangleup C) 
    & = (B \bigtriangleup C) \bigtriangleup A \\
    & = (B \cap C^c \cap A^c) \cup (B^c \cap C \cap A^c) \\
    & \hspace{1cm} \cup (B^c \cap C^c \cap A) \cup (B \cap C \cap A) \\
    & = (A^c \cap B \cap C^c) \cup (A^c \cap B^c \cap C) \\
    & \hspace{1cm} \cup (A \cap B^c \cap C^c) \cup (A \cap B \cap C) \\
    & = (A \cap B^c \cap C^c) \cup (A^c \cap B \cap C^c) \\
    & \hspace{1cm} \cup (A^c \cap B^c \cap C) \cup (A \cap B \cap C) \\
    & = (A \bigtriangleup B) \bigtriangleup C .
  \end{align*}
  最後に式\eqref{eq:symdifcapbunpai}を示す.
  読者は集合$A,  B,  C$に対して$A \cap (B-C) = (A \cap B) - (A \cap C)$
  が成り立つことを示せ.そうすれば,
  \begin{align*}
    A \cap (B \bigtriangleup C) & = A \cap ((B \cup C) - (B \cap C)) \\
                                & = (A \cap (B \cup C) - (A \cap (B \cap C) ) \\
                                & = ((A \cap B ) \cup (A \cap C)) - ((A \cap B) \cap (A \cap C)) \\
                                & = (A \cap B) \bigtriangleup (A \cap C) .
  \end{align*}
\item[\refque{que:setfamilycupcap}] \mbox{} \\
  わからなければ\ref{sec:sequent}を読み返せ.
\item[\refque{que:demorganfamiry}] \mbox{} \\
  述語論理におけるDe Morganの法則を用いればよい.
\item[\refque{que:mappingfamirysubset}] \mbox{} \\
  定理\ref{thm:mapcupcap}と同様にすればよい.
\item[\refque{que:unioninteisectionfamiry}] 
  \begin{align*}
    \bigcup_{n=1}^{\infty} I_n = (-2,2) ,  \bigcap_{n=1}^{\infty} I_n = [-1,1] , \\
    \bigcup_{n=1}^{\infty} J_n = (-1,\infty) ,  \bigcap_{n=1}^{\infty} J_n = [0,1) .
  \end{align*}
\item[\refque{que:limitsetmugenyugen}] \mbox{} \\
  任意の$x$に対し,$x \in E_n$となる$n \in \mathbb{N}$が無限に多く存在することと,
  $x \in E_n$となる$n \in \mathbb{N}$が存在し,さらに
  それに最大のものがないことが同値であること,
  そして$x \notin E_n$となる$n \in \mathbb{N}$が有限個であることと,
  $x \notin E_n$となる$n \in \mathbb{N}$が存在しない,
  もしくはあってもそれに最大のものが存在することが
  同値であることを考えれば明らかであろう.
\item[\refque{que:limitsupinfsubset}] \mbox{} \\
  問\ref{que:limitsetmugenyugen}の結果からほとんど明らかだろう.
\item[\refque{que:limitinfsupsetAB}] \mbox{} \\ 
  式\eqref{eq:limitsupsetcup}に関しては,$x$を任意にとり,
  $x \in A_n \cup B_n$となる$n \in \mathbb{N}$が無限に多く存在することと,
  $x \in A_n$となる$n \in \mathbb{N}$か$x \in B_n$となる
  $n \in \mathbb{N}$の少なくともどちらかは無限に多く存在すること
  が同値であることを用いればよい.
  式\eqref{eq:limitinfsetcap}に関しては,
  $x$を任意にとり,
  $x \notin A_n \cap B_n$となる$ n \in \mathbb{N}$が有限個であることと,
  $x \notin A_n$となる$n \in \mathbb{N}$と$x \notin B_n$となる$n \in \mathbb{N}$
  がともに有限個であることが同値であることを用いればよい.
\item[\refque{que:limitsetzoudaigensyou}] \mbox{} \\
  すべての$n \in \mathbb{N}$に対して$E_n \subset E_{n+1}$が成り立つ場合を考える.
  \begin{align*}
    \liminf_{n \to \infty} E_n 
    & = \bigcap_{k=1}^{\infty} \bigcup_{n=k}^{\infty} E_n \\
    & = \bigcup_{k=1}^{\infty} E_k \\
    & = \bigcup_{n=1}^{\infty} E_n
  \end{align*}
  である.また,任意の$k \in \mathbb{N}$に対して
  \begin{align*}
    \bigcap_{j=1}^{\infty} \bigcup_{n=j}^{\infty} E_n 
    \subset \bigcup_{n=k}^{\infty} E_n \subset \bigcup_{n=1}^{\infty} E_n
  \end{align*}
  が成り立つので,
  \begin{align*}
    \limsup_{n \to \infty} E_n \subset \bigcup_{n=1}^{\infty} E_n
    = \liminf_{n \to \infty}E_n
  \end{align*}
  となる.問\ref{que:limitsupinfsubset}の結果も踏まえれば,
  $\displaystyle \lim_{n \to \infty} E_n $が存在し,
  \begin{align*}
    \lim_{n \to \infty} E_n = \bigcup_{n=1}^{\infty} E_n
  \end{align*}
  となることがわかる.
  すべての$n \in \mathbb{N}$に対して$E_{n+1} \subset E_n$となる場合も同様である.
\item[\refque{que:limitsetcupcap}] \mbox{} \\
  問\ref{que:limitinfsupsetAB}の結果からほとんど明らかである.
\item[\refque{que:limitsetoddeven}] \mbox{} \\
  正の偶数も正の奇数も無限に多く存在することを踏まえれば,
  \begin{align*}
    \limsup_{n \to \infty} E_n & = S \cup T , \\
    \liminf_{n \to \infty} E_n & = S \cap T
  \end{align*}
  となることは容易にわかる.
\item[\refque{que:sumunitele}] \mbox{} \\
  有理数の切断により
  $\alpha = ( A \mid B) ,  0 = (C \mid D),  \alpha + 0 = (E \mid F)$
  と表す.
  $e \in E$を任意にとると,$e= a+c$となる$a \in A ,  c \in C$がとれる.
  $c <0$であるから$e= a+c<a$である.よって$e \in A$であり,
  $E \subset A$が成り立つ.
  次に,$a \in A$を任意にとると,
  $A$には最大数が存在しないので,$a < u$となる$u \in A$がとれる.
  $a-u <0$より$a-u \in C$である.
  ゆえに$a= u + (a-u) \in E$より$A \subset E.$
  従って$A= E$となり,$\alpha + 0 = \alpha$がいえる.
\item[\refque{que:sqrtdef}] \mbox{} \\
  $S$が上に有界な空でない集合であることを示せば$s= \sup S$の存在がわかる.
  あとは$s^2<x$としても$s^2>x$としても矛盾が生じることを示せばよい.
  前者は$s$が$S$の上界であることに,後者は$s$が$S$の上界として
  最小のものであることに矛盾する.
\item[\refque{que:suretuconvsa}] \mbox{} \\
  定理\ref{thm:suretuconvwaseki}の前半と同様にできる.
\item[\refque{que:suretuconvsyou}] \mbox{} \\
  前半部分を示す.$\beta \neq 0$なので,
  $\varepsilon _0 = \lvert \beta \rvert /2$とおくと,
  $\varepsilon _0 >0$となる.
  よって,数列$\{ b_n \}$が$\beta$に収束することから,
  $N \in \mathbb{N}$が存在して,
  $n \geq N$を満たす任意の$n \in \mathbb{N}$について
  \begin{align*}
    \lvert b_n - \beta \rvert < \varepsilon_0 = \frac{ \lvert \beta \rvert }{2}
  \end{align*}
  が成り立つ.これと$\big \lvert \lvert b_n \rvert - \lvert \beta \rvert \big \rvert 
  \leq \lvert b_n - \beta \rvert $より,
  \begin{align*}
    - \frac{ \lvert \beta \rvert }{2} < \lvert b_n \rvert - \lvert \beta \rvert 
    < \frac{ \lvert \beta \rvert }{2} 
  \end{align*}
  だから,左側の不等式をとって
  \begin{align*}
    \lvert b_n \rvert > \frac{ \lvert \beta \rvert }{2}
  \end{align*}
  を得る.

  後半部分を示そう.$\varepsilon >0$を任意にとり,
  \begin{align*}
    \varepsilon _0 = \frac{ \lvert \beta \rvert ^2 }
    {2( \lvert \alpha \rvert + \lvert \beta \rvert )} \varepsilon
  \end{align*}
  とおく.$\beta \neq 0$だから$\varepsilon _0 >0$である.
  さて,数列$\{ a_n \} ,  \{ b_n \}$はそれぞれ$\alpha ,  \beta$に収束するので
  $N_1 ,  N_2 \in \mathbb{N}$が存在して,
  $n \geq N_1$となる任意の$n \in \mathbb{N}$に対して
  $\lvert a_n - \alpha \rvert < \varepsilon _0$が,
  $n \geq N_2$となる任意の$n \in \mathbb{N}$に対して
  $ \lvert b_n - \beta \rvert < \varepsilon_0$がそれぞれ成り立つ.
  さらに,前半部分により,$N_3 \in \mathbb{N}$が存在して,
  $n \geq N_3$となる任意の$n \in \mathbb{N}$に対して
  $\lvert b_n \rvert > \lvert \beta \rvert /2$となる.
  そこで,$N = \max \Set{ N_1 ,  N_2 ,  N_3 }$とおくと,
  $n \geq N$を満たす任意の$n \in \mathbb{N}$に対して
  \begin{align*}
    \left \lvert \frac{a_n}{b_n} - \frac{ \alpha }{\beta} \right \rvert 
    & = \left \lvert \frac{ a_n \beta - \alpha b_n}{b_n \beta} \right \rvert \\
    & = \left \lvert \frac{ a_n \beta + \alpha \beta - \alpha \beta - \alpha b_n}
    {b_n \beta} \right \rvert \\
    & = \left \lvert \frac{ (a_n - \alpha ) \beta - \alpha ( b_n - \beta ) }{b_n \beta }
    \right \rvert \\
    & \leq \frac{ \lvert a_n - \alpha \rvert \lvert \beta \rvert + \lvert \alpha \rvert
    \lvert b_n - \beta \rvert }{\lvert b_n \rvert \lvert \beta \rvert } \\
    & < \frac{ \varepsilon _0 \lvert \beta \rvert + \lvert \alpha \rvert \varepsilon _0}
    { \displaystyle \frac{ \lvert \beta \rvert }{2} \cdot \lvert \beta \rvert } \\
    & = \frac{ 2( \lvert \alpha \rvert + \lvert \beta \rvert )}{\lvert \beta \rvert ^2}
    \varepsilon _0 \\
    & = \varepsilon .
  \end{align*}
  従って,数列$\left \{ \displaystyle \frac{a_n}{b_n} \right \}$は
  $\displaystyle \frac{\alpha }{\beta}$に収束する.
\item[\refque{que:gausskigou}] \mbox{} \\
  $x=0$の場合は$n=0$とおけば$n \leq x < n+1$となる.
  $x>0$の場合を考える.集合$A$を
  \begin{align*}
    A = \Set{ n \in \mathbb{N} \mid x < n }
  \end{align*}
  と定めると,実数のArchimedes性から$A$は空でない$\mathbb{N}$の部分集合である.
  従って,$m= \min A$がとれる.$m$の最小性から,$n=m-1$とおくと$n \notin A$であり,
  $n \leq x <n+1$を満たす.

  $x<0$の場合は$-x>0$であり,$m-1 \leq -x <m$となる整数$m$が存在する.
  $m-1=x$であれば$n=1-m$とおけば$n=x<n+1$より$n \leq x <n+1$が成り立つ.
  $m-1 \neq x$であれば$-m < x < 1-m$であるから$n=-m$とおけば
  $n \leq x < n+1$が成り立つ.

  あとは一意性を示すだけである.$n ,  m \in \mathbb{Z}$を
  $n \leq x < n+1 ,  m \leq x < m+1$が成り立つよう任意にとる.
  $n<m$と仮定すると,$n,  m$は整数であるから$n+1 \leq m$となり,
  $x<n+1 \leq m$となり,$m \leq x$に矛盾する.
  $m<n$と仮定しても同様に矛盾するので$n=m$である.
\item[\refque{que:bekisyusoku}] \mbox{} \\
  式\eqref{eq:n1syusoku}を示す.
  $\mathbb{R}$のArchimedes性により,任意の$\varepsilon > 0$に対して
  $N \varepsilon >1$となる$N \in \mathbb{N}$が存在する.
  このとき$1/N < \varepsilon$である.
  よって,$n \geq N$となる任意の$n \in \mathbb{N}$に対して
  \begin{align*}
    \left \lvert \frac{1}{n} - 0 \right \rvert = 
    \frac{1}{n} \leq \frac{1}{N} < \varepsilon
  \end{align*}
  であるから$\displaystyle \lim_{n \to \infty} \frac{1}{n} =0$である.

  次に式\eqref{eq:r1syusoku}を示す.$0<r<1$より$ 1< 1/r$
  であるから$1/r = 1+h$とおけば$h>0$.
  よって,二項定理から
  \begin{align*}
    0<  r^n & = \frac{1}{(1+h)^n} \\
            & = \frac{1}{ 1+ nh + \displaystyle \frac{n(n-1)}{2}h^2 + \cdots + h^n} \\
            & < \frac{1}{nh}
  \end{align*}
  となる.
  $\displaystyle \lim_{n \to \infty} \frac{1}{nh}=0$であるから,
  はさみうちの原理により$\displaystyle \lim_{n \to \infty} r^n = 0$となる.




\item[\refque{que:finitecardAB}] \mbox{} \\
  $\lvert A \rvert = n ,  \lvert B \rvert = m$とおき,
  定理\ref{thm:yugeninjsurj}と定理\ref{thm:cardinalwelldef}
  を利用せよ.なお,$n,  m$が0となるような場合には
  空写像の性質を思い出す必要がある.


\item[\refque{que:dedekindiniffini}] \mbox{} \\
  補題\ref{lemma:subsetfini}の証明を見れば,
  有限集合$A$に対し,集合$B$が$B \subsetneq A$を満たすのであれば
  $B$は有限集合であり,かつ$\lvert B \rvert < \lvert A \rvert$
  となることは容易にわかる.
\item[\refque{que:houjogenri}] \mbox{} \\
  $n$に関する帰納法による.1個の有限集合に対しては明らか.
  いま,$n$個の有限集合に対しては
  主張が正しいとして,$n+1$個の有限集合
  $A_1, A_2, \ldots , A_n , A_{n+1}$について考えると,
  \begin{align*}
    \bigcup_{i=1}^{n+1} A_i = \left( \bigcup_{i=1}^{n} A_i \right) \cup A_{n+1}
  \end{align*}
  であるから$A_1, A_2, \ldots A_n , A_{n+1}$の和集合も有限集合であって,

  \begin{align*}
    \left \lvert \bigcup_{i=1}^{n+1} A_i \right \rvert 
    & =\left \lvert \left( \bigcup_{i=1}^{n} A_i \right) \cup A_{n+1} 
      \right \rvert \\
      \allowdisplaybreaks[4]
    & = \left \lvert \bigcup_{i=1}^{n} A_i \right \rvert 
      + \lvert A_{n+1} \rvert - 
      \left \lvert \left( \bigcup_{i=1}^{n} A_i \right) \cap A_{n+1} \right \rvert \\
    & = \sum_{j=1}^{n} (-1)^{j-1} \sum_{1 \leq k_1 < \cdots < k_j \leq n }
      \lvert A_{k_1} \cap \cdots \cap A_{k_j} \rvert  \\
    &  \hspace{1.5cm} + (-1)^{1-1} \lvert A_{n+1} \rvert 
      + \left \lvert \bigcup_{i=1}^{n} ( A_i \cap A_{n+1} ) \right \rvert \\
    & = \sum_{j=1}^{n} (-1)^{j-1} \sum_{1 \leq k_1 < \cdots < k_j \leq n }
      \lvert A_{k_1} \cap \cdots \cap A_{k_j} \rvert  \\
    &  \hspace{0.2cm} + (-1)^{1-1} \lvert A_{n+1} \rvert \\
    &  \hspace{0.4cm} \sum_{j=1}^{n} (-1)^{j-1} \\
    &  \hspace{0.6cm}  \sum_{1 \leq k_1 < \cdots < k_j \leq n} 
      \lvert (A_{k_1} \cap A_{n+1} ) \cap \cdots \cap (A_{k_j} \cap A_{n+1}) \rvert \\ 
    & = \sum_{j=1}^{n} (-1)^{j-1} \sum_{1 \leq k_1 < \cdots < k_j \leq n }
      \lvert A_{k_1} \cap \cdots \cap A_{k_j} \rvert  \\
    &  \hspace{0.3cm} + (-1)^{1-1} \lvert A_{n+1} \rvert \\
    &  \hspace{0.6cm} + \sum_{j=1}^{n} (-1)^{j-1} \\
    &  \hspace{0.9cm}  \sum_{1 \leq k_1 < \cdots < k_j \leq n} 
      \lvert (A_{k_1} \cap \cdots \cap A_{k_j} ) \cap A_{n+1} \rvert \\
      & = \sum_{j=1}^{n+1}(-1)^{j-1} 
      \sum_{1 \leq k_1 < \cdots < k_j \leq n+1} 
      \lvert A_{k_1} \cap \cdots \cap A_{k_j}   \rvert .
  \end{align*}

\item[\refque{que:casansubcasan}] \mbox{} \\
  可算集合$A$に対し,$A$の無限部分集合$B$を考える.
  $\lvert A \rvert = \aleph _0$であるから
  全単射$f: A \longrightarrow \mathbb{N}$がとれる.
  $b \in B$に対し,$f(b)$以下であるような
  自然数の個数は有限で,しかも一意に定まる.
  この個数を$g(b)$として
  写像$g: B \longrightarrow \mathbb{N}$を定めると,
  $g$は全単射である.従って$\lvert B \rvert = \aleph _0$となる.


\item[\refque{que:AB2casan}] \mbox{} \\
  $\lvert A \rvert = \lvert B \rvert = \aleph _0$だから,
  2つの全単射$f: A \longrightarrow \mathbb{N} , 
  g: B \longrightarrow \mathbb{N}$がとれる.
  そこで,写像$h: A \times B \longrightarrow \mathbb{N}$を
  \begin{align*}
    h(a,b) & = \frac{1}{2} ( f(a) + g(b) -1)( f(a) + g(b) -2) + g(b) \\
           & \quad ( (a,b) \in A \times B )
  \end{align*}
  と定義すると,$h$は全単射である.
  よって,$\lvert A \times B \rvert = \aleph _0 .$

\item[\refque{que:casan2tyoku}] \mbox{} \\
  まずは$n \in \mathbb{N}$に対して
  写像$F : \mathbb{N} \times \Set{ 1,2, \ldots , n} \longrightarrow \mathbb{N}$を
  \begin{align*}
    F ( i, j ) = n(i-1) + j \quad ( ( i,j) \in \mathbb{N} \times \Set{ 1,2, \ldots , n})
  \end{align*}
  と定めたとき,$F$が全単射であることを示せ.
  単射性に関しては,
  $(i,j) \in \mathbb{N} \times \Set{1,2,\ldots , n}$に対して
  $F(i,j)$を$n$で割った余りに着目せよ.
  全射性に関しては数学的帰納法を用いよ.

  さて,$\lvert A \rvert = \aleph _ 0$より全単射$f: A \longrightarrow \mathbb{N}$
  がとれて,$\lvert B \rvert = n$とおくと
  全単射$g: B \longrightarrow \Set{ 1,2, \ldots , n}$がとれるから,
  写像$h: A \times B \longrightarrow \mathbb{N}$を
  \begin{align*}
    h(a,b) = n ( f(a)-1) + g(b) \quad ( (a,b) \in A \times B)
  \end{align*}
  と定めると,$h$は全単射である.
  従って$\lvert A \times B \rvert = \aleph _0 .$

\item[\refque{que:soejikasan}] \mbox{} \\
  各$A_{\lambda}$がすべて空であるならばその和集合も空であり,高々可算集合である.
  各$A_{\lambda}$のうち空でないものが少なくとも1つあるとき,
  そのような$\lambda$全体の集合を
  $\varLambda '$とすると,$\varLambda '$は空でない高々可算集合であり,
  $(A_{\lambda})_{\lambda \in \varLambda}$の和集合と$(A_{\lambda})_{\lambda \in \varLambda '}$
  の和集合は等しい.従って,各$A_{\lambda}$がすべて空でない場合を考えればよい.

  $\varLambda$は空でない高々可算集合であるから,$\varLambda \times \mathbb{N}$
  は可算集合である.従って,全単射$f: \mathbb{N} \longrightarrow \varLambda \times \mathbb{N}$
  が存在する.さて,各$\lambda \in \varLambda$に対し,$A_{\lambda}$は高々可算集合だから,
  単射$g_{\lambda} : A_{\lambda} \longrightarrow \mathbb{N}$が存在する.$A_{\lambda}$
  は空でないから,全射$h_{\lambda} : \mathbb{N} \longrightarrow A_{\lambda}$が存在する.
  そこで,写像$\varphi : \varLambda \times \mathbb{N} \longrightarrow 
  \bigcup_{\lambda \in \varLambda } A_{\lambda}$を
  \begin{align*}
    \varphi ( \lambda , n ) = h_{\lambda} (n) \quad
    ( ( \lambda , n) \in \varLambda \times \mathbb{N})
  \end{align*}
  と定義すると,$\varphi$は全射である.
  よって,合成写像$\varphi \circ f: \mathbb{N} \longrightarrow 
  \bigcup_{\lambda \in \varLambda} A_{\lambda}$
  は全射である.$\bigcup_{\lambda \in \varLambda} A_{\lambda}$は空でないから,
  $\bigcup_{\lambda \in \varLambda} A_{\lambda}$から$\mathbb{N}$への単射が存在する.
  ゆえに$\bigcup_{\lambda \in \varLambda} A_{\lambda}$は高々可算集合である.

\item[\refque{que:mugencasan}] \mbox{} \\
  $M$は無限集合であるから,定理\ref{thm:casansubchoice}により
  $M$の部分集合$B$で可算集合であるものがとれる.
  $A$が高々可算集合なので,$A \cup B$は可算集合である.
  従って,全単射$f: A \cup B \longrightarrow B$
  が存在する.
  そこで,写像$g: M \cup A \longrightarrow M$を
  \begin{align*}
    g(x) = \left \{
      \begin{aligned}
        f(x) \quad & ( x \in A \cup B \text{のとき} ) , \\
        x \qquad & ( \text{それ以外のとき} )
      \end{aligned}
      \right.
  \end{align*}
  と定めると,$g$は全単射である.
  従って,$\lvert M \cup A \rvert = \lvert M \rvert$となる.

\item[\refque{que:nyugenkasan}] \mbox{} \\
  \begin{enumerate}
    \item 非負の整数$n$に対し,
      $\mathbb{N}$の部分集合で濃度が$n$であるような
  ものの全体の集合を$A_n$とおくと,各$A_n$はすべて有限集合で,
  \begin{align*}
    \mathscr{S}_1 = \bigcup_{n=0} ^ { \infty} A_n
  \end{align*}
  が成り立つ.このことと問\ref{que:soejikasan}の結果により,
  $\mathscr{S}_1$が高々可算集合であることがわかる.
  $\mathscr{S}_1$は明らかに無限集合だから,
  $\lvert \mathscr{S}_1 \rvert = \aleph _0$となる.


\item $\mathscr{S}_1 \cup \mathscr{S} = \mathfrak{P}
  ( \mathbb{N} )$であり,
  定理\ref{thm:bekinoudo}から$\lvert \mathbb{N} \rvert < 
  \lvert \mathfrak{P} ( \mathbb{N} ) \rvert$
  である.
  $\mathscr{S}_1$が可算集合であることも踏まえると,
  $\mathscr{S}$が無限集合でなければならないことが分かる.
  ゆえに問\ref{que:mugencasan}の結果から
  $\lvert \mathscr{S} \rvert = \lvert \mathscr{S}_1 
  \cup \mathscr{S} \rvert = \lvert \mathfrak{P}
  ( \mathbb{N} ) \rvert$を得る.

\item 無限小数展開の存在と一意性から,
  写像$f: \mathscr{S} \longrightarrow (0,1]$を
  \begin{align*}
    f(N) = \sum_{n=1}^{\infty} \frac{\chi_N ( n) }{2^n} \quad ( N \in \mathscr{S} )
  \end{align*}
  と定めることができ,$f$は全単射である.
  ここで,$\chi_N : \mathbb{N} \longrightarrow \Set{0,1}$は
  $\mathbb{N}$における$N \in \mathscr{S}$の指示関数である.
  これにより,
  $\lvert \mathscr{S} \rvert = \left \lvert (0,1] \right \rvert $であることがわかる.
  $\lvert \mathscr{S} \rvert = \lvert \mathfrak{P}(\mathbb{N} ) \rvert$と
  $\left \lvert (0,1] \right \rvert = \aleph$により
  $\lvert \mathfrak{P}(\mathbb{N})\rvert = \aleph$を得る.
  \end{enumerate}

\item[\refque{que:heiheikaikai}] \mbox{} \\
  写像$f: [a,b] \longrightarrow [c,d] $を
  \begin{align*}
    f(x) = \frac{d-c}{b-a} ( x-a) +c \quad ( x \in [a,b] )
  \end{align*}
  と定めると,$f$は全単射である.
  $f$の始集合と終集合を開区間にとりかえれば,
  開区間$(a,b)$から開区間$(c,d)$への全単射もつくれる.

\item[\refque{que:heikaizen}] \mbox{} \\
  写像$h: [0,1] \longrightarrow (0,1) $を
  \begin{align*}
    h(x) = \left \{
      \begin{aligned}
        \frac{2x+1}{4} \quad & \left ( x = \frac{1}{2} \pm 
        \left ( \frac{1}{2} \right) ^n \, ( n \in \mathbb{N} ) \text{のとき} \right) , \\
        x \qquad & ( \text{それ以外のとき} )
      \end{aligned}
      \right.
  \end{align*}
  と定義すると,$h$は全単射である.

\item[\refque{que:mugenhankai}] \mbox{} \\
  写像$h: (0, \infty) \longrightarrow (0, 1]$を
  \begin{align*}
    h(x) = \left \{
      \begin{aligned}
        x \qquad & \left( x= \frac{1}{n} \, ( n \in \mathbb{N} ) \text{のとき} ) \right) , \\
        \frac{x}{x+1} \quad & ( \text{それ以外のとき} )
      \end{aligned}
      \right.
  \end{align*}
  と定めると,$h$は全単射である.

\item[\refque{que:R2Rnoudo}] \mbox{} \\
  \begin{enumerate}
    \item $x \in \mathbb{R}$に対して$(x,x) \in \mathbb{R}^2$を対応させる写像を考えると,
      これは明らかに単射である.
    \item 写像$g_1 : \mathbb{R} \longrightarrow (0,1)$を
      \begin{align*}
        g_1(x) = \frac{1}{2} (f(x) + 1 ) = \frac{1+ \lvert x \rvert +x}{2(1+ \lvert x \rvert)}
        \quad ( x \in \mathbb{R} )
      \end{align*}
      と定めると,$g_1$は全単射である.そこで,写像$g: \mathbb{R}^2 \longrightarrow D$を
      \begin{align*}
        g(x,y) 
        = \left( \frac{1+ \lvert x \rvert + x } {2(1+ \lvert x \rvert ) } , 
        \frac{1+ \lvert y \rvert +y }{2(1+ \lvert y \rvert )} \right)
        \quad ( (x,y) \in \mathbb{R}^2 )
      \end{align*}
      と定めると,$g$は$\mathbb{R}^2$から$D$への全単射である.
    \item $x, y \in D$を10進展開して
      \begin{align*}
        x & = 0. x_1 x_2 x_3 \cdots \\
        y & = 0. y_1 y_2 y_3 \cdots 
      \end{align*}
      と表す.無限小数展開の存在と一意性により,
      写像$h: D \longrightarrow \mathbb{R}$を
      \begin{align*}
        h(x,y) = 0.x_1 y_1 x_2 y_2 x_3 y_3 \cdots \quad ( (x,y) \in D ) 
      \end{align*}
      と定めることができ,$h$は単射である.
     \item (1)より$\lvert \mathbb{R} \rvert \leq                                                       
      \left \lvert \mathbb{R}^2 \right \rvert$である.
      $(2),(3)$より$\left \lvert \mathbb{R}^2 \right \rvert = \lvert D \rvert 
      \leq \lvert \mathbb{R} \rvert$だから$\left \lvert \mathbb{R}^2 \right \rvert 
      \leq \lvert \mathbb{R} \rvert $となる.
      従って,Bernsteinの定理より,
      $\left \lvert \mathbb{R} ^2 \right \rvert = \lvert \mathbb{R} \rvert$となる.
  \end{enumerate}

\item[\refque{que:algnum}] \mbox{} \\
  与えられた代数方程式\eqref{eq:daisuuhouteisiki}に対し,
  $h= \lvert a_n \rvert + \lvert a_{n-1} \rvert + \cdots + \lvert a_0 \rvert + n$
  とおき,これを代数方程式\eqref{eq:daisuuhouteisiki}の高さと呼ぶことにする.
  さて,高さ$m$の代数方程式は有限個である.代数学の基本定理により,
  次数$m$の代数方程式の複素数解の個数は高々$m$個である.
  従って,高さ$m$の代数方程式の解となっている代数的数は有限個である.
  ゆえに,高さ$m$の代数方程式の解となっている代数的数全体の集合を
  $A_m$とすると,各$A_m$はすべて有限集合である.
  よって,$\overline{\mathbb{Q}} = \bigcup_{m=2}^{\infty} A_m$は
  高々可算集合である.$\mathbb{Q} \subset \overline{\mathbb{Q}}$だから
  明らかに$\overline{\mathbb{Q}}$は無限集合なので,
  結局$\overline{\mathbb{Q}}$は可算集合であることがわかる.



\item[\refque{que:quohuhen}] \mbox{} \\
  $\tilde{f}(A) = \tilde{f}(B)$を満たす$A,B \in X/{\sim}$を任意にとり,
  $A,B$を$x,y \in X$を用いて$A= [x] , B=[y]$と表す.
  $\tilde{f} (A) = f(x) , \tilde{f} (B) = f(y)$だから,$\tilde{f} (A) = \tilde{f}(B)$
  より$f(x) = f(y)$となる.よって$x \sim y$となるから$[x]= [y]$となる.
  従って$A=B$が成り立つから,$\tilde{f}$は単射である.



\item[\refque{que:groupdouti}] \mbox{} \\
  反射律は$G$が恒等写像を元にもつこと,
  対称律は$G$の元すべてに対してその元の逆写像も
  $G$の元にもっていること,推移律は$G$の元同士の合成写像もまた
  $G$の元になっていることからそれぞれわかる.

\item[\refque{que:cauchyretu}] \mbox{} \\
  反射律は明らか.対称律は絶対値の対称性からいえる.
  推移律に関しては,三角不等式を用いればよい.

\item[\refque{que:kyokusai}] \mbox{} \\
  $S$に最大元$a$が存在する場合を考える.
  $x \in S$を任意にとると,
  $a$は$S$の最大元であるから$x \leq a$が成り立つ.
  従って,$a<x$となるような$x \in S$は存在しない.
  ゆえに$a$は$S$の極大元である.
  また,$a$でない$S$の任意の元$a'$に対し,つねに
  $a' < a$となることから,$a'$は$S$の極大元とはなりえない.
  ゆえに,$a$のみが$S$の極大元である.
  $S$に最小元が存在する場合も同様である.
  

\item[\refque{que:NZQRdoukei}] \mbox{} \\
  $\lvert \mathbb{N} \rvert = \lvert \mathbb{Z} \rvert 
  = \lvert \mathbb{Q} \rvert < \lvert \mathbb{R} \rvert$
  であるから,$\mathbb{R}$は他のいずれとも順序同型でない.
  また,$\mathbb{N}$には最小元が存在するが,
  $\mathbb{Z}$と$\mathbb{Q}$には最小元は存在しないため,
  $\mathbb{N}$は$\mathbb{Z}, \mathbb{Q}$のいずれとも順序同型でない.
  $\mathbb{Z}$と$\mathbb{Q}$が順序同型であるとし,
  順序同型写像$f: \mathbb{Z} \longrightarrow \mathbb{Q}$をとる.
  $0<1$であるから$f(0) < f(1)$でなければならない.
  $c= (f(0) + f(1))/2$とおくと$f(0) < c < f(1)$となり,
  $f^{-1}$も順序を保つから$0<f^{-1}(c)< 1$とならなくてはならない.
  しかし,このような整数$f^{-1}(c)$は存在しないので矛盾である.
  ゆえに,$\mathbb{Z}$と$\mathbb{Q}$は順序同型でない.

  また,$\mathbb{Z}$上に順序$\leq$を,$x , y \in \mathbb{Z}$に対して
  $\lvert x \rvert \leq \lvert y \rvert $ならば$x \leq y$とし,
  $\lvert x \rvert = \lvert y \rvert$の場合には,$x$が正である場合には
  $x \leq y$とし,それ以外の場合には$y \leq x$として定義する.
  $0 \leq 1 \leq -1 \leq 2 \leq -2 \leq \cdots $のようになり,
  この順序を入れた$\mathbb{Z}$は$\mathbb{N}$と順序同型となる.


\item[\refque{que:Rkaikukanjunjo}] \mbox{} \\
  $\mathbb{R}$上の任意の開区間$(a,b)$に対し,
  写像$f: (a,b)  \longrightarrow \mathbb{R}$を
  \begin{align*}
    f(x) = \frac{1}{a-x} + \frac{1}{b-x} \quad ( x \in (a,b) )
  \end{align*}
  と定めると,$f$は順序同型写像である.

\item[\refque{que:doukeigyaku}] \mbox{} \\
  たとえば,集合$\set{a_1,a_2,a_3} ,
  \set{b_1,b_2,b_3}$
  上の順序$\leq, \leq'$をそれぞれ
  \begin{align*}
    \leq & = \set{ (a_1,a_2) , (a_1,a_3)} \\
    \leq' & = \set{ (b_1,b_2) , (b_1,b_3) , (b_2,b_3)}
  \end{align*}
  と定めれば,$f$は順序を保つが$f$の逆写像
  $f^{-1}$は順序を保たない.



\item[\refque{que:Nseiretu}] \mbox{} \\
  $N$を$\mathbb{N}$の空でない部分集合とする.
  任意の自然数$n$に対し,
  $N$の元で$n$以下のものが存在すれば
  $N$には必ず最小元が存在することを帰納法によって示す.
  $1 \in N$であれば明らかに1が$N$の最小元である.
  $N$の元で$n$以下ものが存在するならば
  $N$には最小元が存在すると仮定して,
  $N$の元で$n+1$以下のものが存在するならば
  $N$には最小元が存在することを示す.
  $N$の元で$n$以下のものが存在する場合,
  帰納法の仮定によって$N$には最小元が存在する.
  $N$の元で$n$以下のものが存在しない場合,
  任意の$m \in N$に対して$n+1 \leq m$が成り立つため,
  $n+1$が$N$の最小元である.
  
  さて,$N$は空でないから,$m \in N$が存在する.
  このとき$m+1 \in \mathbb{N}$であり,
  $N$は$m+1$以下の元をもつ.
  ゆえに$N$は最小元をもつ.


\item[\refque{que:prozensya}] \mbox{} \\
  $\lambda \in \varLambda$を任意にとり,
  $x \in A_{\lambda}$を任意に1つとって固定しておく.
  選択公理により,$f \in \prod_{\lambda \in \varLambda} A_{\lambda}$
  が1つとれる.
  そこで,
  写像$g: \varLambda \longrightarrow \bigcup_{\lambda \in \varLambda} A_{\lambda}$を
  \begin{align*}
    g ({\mu} )= \left \{
      \begin{aligned}
         f_{\mu} \quad & ( \mu \neq \lambda \text{のとき} ) , \\
        x \quad \, & (\mu = \lambda \text{のとき}) 
      \end{aligned}
      \right.
  \end{align*}
  と定めると,$g \in \prod_{\lambda \in \varLambda} A_{\lambda}$
  であって$p_{\lambda} (g) = x$となる.
  ゆえに$p_{\lambda}$は全射である.

\item[\refque{que:dedeinf}] \mbox{} \\
  無限集合$A$に対し,可算選択公理から$A$の部分集合$B$で
  可算集合であるものがとれる.$A$が可算集合であれば,
  全単射$f : \mathbb{N} \longrightarrow A$
  が存在する.また,$2 \mathbb{N} 
  = \Set{ 2n \mid n \in \mathbb{N} }$
  は可算集合なので,$\lvert \mathbb{N} \rvert 
  = \lvert 2 \mathbb{N} \rvert$である.そこで,
  写像$g: f(2 \mathbb{N}) \longrightarrow A$を
  $g(f(2n)) = f(n) \ (n \in \mathbb{N})$
  と定めれば,$g$は全単射である.
  従って,$\lvert f(2 \mathbb{N} ) \rvert = \lvert A \rvert$
  となり,$f(2 \mathbb{N})$は$A$の真部分集合だから
  $A$はDedekind無限集合である.
  $A$が可算集合でない場合には,
  $A-B$は無限集合である.
  実際,$A-B$が有限集合であるとすると,
  $(A- B) \cup B = A$が可算集合となる.
  ゆえに,
  問\ref{que:mugencasan}の結果から
  $\lvert A \rvert = \lvert (A-B)  \cup B \rvert 
  = \rvert A-B \rvert$となり,$A-B$は$A$の真部分集合だから
  $A$はDedekind無限集合である.

  

\item[\refque{que:chap4_hikaku}] \mbox{} \\
  与えられた集合$A,B$に対し,
  整列可能定理より,$(A, {\leq_A}), (B, {\leq_B})$
  が整列集合になるような順序$\leq_A, \leq_B$が存在する.
  さらに,$A$と$B$が
  順序同型であるか,$A$が$B$の切片と順序同型であるか,
  あるいは$B$が$A$の切片と順序同型であるかの
  いずれか1つだけが必ず成り立つ.
  このとき,$\lvert A \rvert =
  \lvert B \rvert , \lvert A \rvert < \lvert B \rvert , 
  \lvert A \rvert > \lvert B \rvert$
  がそれぞれ対応して成り立つ.

\item[\refque{que:chap4_tukey}] \mbox{} \\
  Zornの補題からTukeyの補題を導くことを考える.
  包含関係によって順序を入れた
  有限性を満たす空でない集合族$\mathscr{F}$について,
  その全順序部分集合$\mathscr{B}$を任意にとり,
  $A = \bigcup \mathscr{B}$とおく.
  $A$の有限部分集合$B = \Set{ b_1, b_2, \ldots, b_n}$
  を任意にとると,各$b_i$に対して
  $b_i \in C_i$となるような$C_i \in \mathscr{B}$
  が存在する.
  このとき,$B \subset \bigcup_{i=1}^{n} C_i$となる.
  $\mathscr{B}$は全順序集合であるから,
  $C= \max \Set{ C_i \mid i= 1, 2, \ldots , n}$
  がとれて,この$C$は$B \subset C \in \mathscr{F}$
  を満たし,$\mathscr{F}$が有限性を満たすことから
  $B \in \mathscr{F}$となることがわかる.
  さらに$\mathscr{F}$の有限性により
  $A \in \mathscr{F}$となり,
  $A$が$\mathscr{B}$の上界であることがわかる.
  ゆえに$\mathscr{F}$は順序集合として帰納的であって,
  Zornの補題により極大元をもつ.

  次に,Tukeyの補題からZornの補題を導く.
  帰納的な順序集合$(X, \leq)$に対し,
  $X$の部分集合で全順序集合であるものの全体の
  集合を$\mathscr{X}$とおき,$\mathscr{X}$に
  包含関係によって順序を入れる.
  $\varnothing \in \mathscr{X}$であるから$\mathscr{X}$
  は空でない.
  さて,$A \in \mathscr{X}$を任意にとると,
  $A$の任意の有限部分集合は全順序集合であるから,
  それらはすべて$\mathscr{X}$の元である.
  逆に,任意の集合$A$に対し,
  その有限部分集合がすべて$\mathscr{X}$の元であるとすると,
  任意の$a, b \in A$に対して$\Set{ a, b } \in \mathscr{X}$
  であるから,集合$\Set{ a,b}$は全順序集合であって,
  $a \leq b$か$b \leq a$のいずれかは必ず成り立つ.
  ゆえに$A$も全順序集合であって$A \in \mathscr{X}$
  となり,$\mathscr{X}$は有限性を満たすことがわかる.
  従って,Tukeyの補題から$\mathscr{X}$の
  極大元$X_0$がとれる.$X_0$は全順序集合であるから
  その上界$x_0$が存在する.
  この$x_0$が$X$の極大元であることを示そう.
  $x_0 < w$となる$w \in X$が存在したとすると,
  $X_0$は集合$X_0 \cup \Set{ w }$の真部分集合である.
  $X_0$は全順序集合で,$x_0 < w$が成り立つことから,
  推移律によってすべての$x \in X_0$に対して
  $x \leq w$が成り立つ.
  ゆえに$X_0 \cup \Set{w}$も全順序集合であって,
  $X_0$が$\mathscr{X}$の極大元であることに矛盾する.
  従って,$x_0$は$X$の極大元である.

\item[\refque{que:senkeikitei}] \mbox{} \\
  $V$の部分集合のうち,1次独立であるものの全体の集合を
  $\mathscr{B}$とし,$\mathscr{B}$に包含関係によって順序を入れる.
  $0$でない$v \in V$を1つとり,集合$\set{ v}$を考えると,
  これは1次独立であるから$\mathscr{B}$は空でなく,
  さらに$\mathscr{B}$は明らかに有限性を満たす.
  従ってTukeyの補題から,$\mathscr{B}$
  の極大元$B$が存在する.
  $B$が$V$の生成系であることを示そう.
  $B$の有限個の元の線型結合として表せない元$b \in V$が
  存在したとすると,$b \notin B$で,
  かつ集合$B \cup \Set{ b}$は1次独立である.
  これは,$B$が$\mathscr{B}$の極大元であることに矛盾する.
  従って$B$は$V$の1次独立な生成系,
  すなわち$V$の基底である.






\end{description}
